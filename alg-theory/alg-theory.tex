\chapter{Distributed transactions}\label{chapter:distributedTransactions}

The method presented in this chapter for indexing graphs in a hybrid way is a solution that solves the problems with
encoding, encountered with the methods presented in the previous chapter.  

\section{Data structures}\label{sec:basics:dataStructures}

Data structures we are discussing in this work may be viewed in the contexts of their applications and theory --
especially the lattice theory and the graph theory. We may consider different issues connected with management and
utilization of these structures, in all of the areas mentioned above. Therefore, we will sketch these contexts in the
next paragraphs. We will also set some terms that we will use in the rest of the work.

\subsection{The area of applications}
In the area of applications, the studied data structures are commonly used as hierarchies of different types, especially
as inheritance hierarchies. A characteristic feature of them is an existence of a special element, called a root. The
rest of elements are connected to the root directly or through other elements. Depending on what sort of hierarchy we
are exploring, various sets of terms are possible to describe specific elements. For example, in case of employees
hierarchy, each element represents an employee. An employee's chief is its predecessor in the hierarchy whilst
employee's subordinates are its successors. In the area of object-oriented programming languages and ontologies,
predecessors and successors are called super-classes and sub-classes respectively \footnote{In the area of ontologies
there are yet other terms that are frequently used. They are: concept, super-concept and sub-concept.}, whereas a
connection between a pair of elements is called inheritance or ``is-a'' relation. Another terminology is  depicted on
Fig. \ref{fig:hierarchyElements}. \cite{CassandraDataStaxDocs} \cite{chandra2007PaxosMadeLive} \cite{lamport1982byzantine}