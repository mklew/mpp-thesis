wyciete-z-chapter3.tex

% Wycięte fragmenty z TransactionState
Transaction id identifies the transaction and is attached to all subsequent requests. 

Transaction \transaction is identified using \emph{unique id} (UUID\footnote{universal unique identifier}), which is generated by \node{i} upon request to begin transaction. 

There are many concurrent transactions, therefore transaction state should use least amount of memory possible, therefore tokens are used instead of keys. Existence of a \txItem means that replicas identified by the \emph{token} store modification of the key done in \transaction. Keys are unknown to the transaction state, but keys are known to the replicas. Each replica has full information about the key and operations performed on the value associated with it. 

Transaction item includes also a table name in order to distinguish modifications associated with the same token, but in different tables. This can be extended to any other name spacing.
Token alone only directs us to nodes with modified data, but to know what was modified, we also need names of tables. 

Transaction state has deterministic memory footprint, since it stores fixed-length tokens instead of keys with unknown size. 


\subsubsection{Alternative structure}
Transaction state which stores keys instead of tokens provides the same functionality, but varies in memory footprint,
hence \mpt is applicable to \emph{key-value} databases in general, but its memory usage might not be as low as with tokens if keys have more complex type than number.
%Same functionality could be achieved storing real keys in transaction items, but then each key could have arbitrary
size and the same would apply to transaction state.

\subsubsection{Unique id only}
Transaction items identify nodes that own part of transaction's data. Without them we would know only an id of a transaction.
Using just an id we would need to ask each node in the cluster if it has data associated with transaction. As a corollary the algorithm would be dependent on all nodes in the cluster, thus resilience to partial failure would be lost.

\begin{definition}
  \label{def:transactionItem}
  \emph{Transaction item}, denoted as $ti$, is a pair of a token and a table name $(tk,table)$.
\end{definition}

\begin{definition}
  \label{def:transactionState}
  \emph{Transaction state}, denoted as $TS$, is a pair of a unique transaction id and set of transaction items $(id, \{ \text{\txItemi{a}, ..., \txItemi{k}} \})$.
\end{definition}


% Stary algorytm

\begin{algorithm}
  \caption{Updating transaction state after two mutations}
  \label{alg:updateTxState}
  \begin{algorithmic}  	    
    \State $TS(id, \{ \}) \gets$ \Call{Begin Transaction}{}
    \State $id \gets$ $TS.id$ unique id generated for transaction
    \State $ti_{1} \gets $ \Call{Update Operation}{id, $k_{1}$, $v_{1}$} \Comment for some key and value
    \State $TS(id, \{ ti_{1} \}) \gets$ \Call{Add Item}{$TS$, $ti_{1}$}
    \State $ti_{2} \gets $ \Call{Insert Operation}{id, $k_{2}$, $v_{2}$}
    \State $TS(id, \{ ti_{1}, ti_{2} \}) \gets$  \Call{Add Item}{$TS$, $ti_{2}$}
    \State $result \gets $ \Call{Commit Transaction}{$TS$}    
  \end{algorithmic}
\end{algorithm}

% Private Transaction Storage

\begin{lstlisting}[language=Java,style=outcode,label={lst:privateTransactionStorage},caption={API of private transaction storage}]
PrivateTransactionStorage {
  TransactionItem addModification(transaction id, modification)

  void removeTransactionsData(transaction id)

  Collection<Modification> getModifications(transaction id)        
}

\end{lstlisting}

\emph{Modification} returned by \code{getModifications} method is Cassandra's representation of any modification operation, such as: \begin{enumerate*}[label=\alph*)] \item insert, \item update, \item delete. \end{enumerate*}

Private transaction storage satisfies requirement of performing many operations within \transactionj. Each \mutation{k}{v} is stored in the private storage and mapped to a \txItem.



% Transaction Log
The log stores information off memory, in dedicated table, in order to preserve information about transaction after it is removed from \emph{Transaction Index}.
