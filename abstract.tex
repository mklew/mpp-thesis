%!TEX root = thesis.tex
\newpage
\paragraph{Abstract} Multiple-inheritance hierarchies are data structures of a high importance in the area of
the artificial intelligence. The problem with them is lack of comprehensive and efficient methods of indexing, which
limits the practical applications. The proposed hybrid indexing method can adapt to the particular topology of the data
structure being processed. We have achieved this by tuning the proportion between the member encoding methods to gain
compactness of the index and fast responses of the search requests. Moreover, we have found that combining a few methods
together preserves and even improves the ability to encode the changes incrementally.

% We have devised two member encoding methods: the first that bases on inheritance of some features and the second that
% bases on numbering schemes. The correctness of the developed solutions has been proved and their performance has been
% evaluated in the testing environment. The results show the characteristics of the hybrid index and prove its
% efficiency.

\paragraph{Keywords:} indexing transitive relations, multiple-inheritance hierarchy, reachability testing

\begin{center}
\vspace*{\baselineskip}
    \fontseries{b}\fontsize{15pt}{18pt}\selectfont
      Transakcje dla wielu partycji w Cassandrze  \\
\end{center}
\paragraph{Streszczenie} Hierarchie dziedziczenia wielokrotnego są strukturami danych o szczególnym znaczeniu w
obszarze zastosowań sztucznej inteligencji. Problemem, który ogranicza ich wykorzystanie w praktyce jest brak
uniwersalnej i efektywnej metody ich indeksowania. Zaproponowana hybrydowa metoda indeksowania dostosowuje się do
topologii przetwarzanej struktury danych. Osiągnięto to po przez dostrajanie proporcji pomiędzy składowymi metodami
kodowania, tak aby indeks był kompaktowy i oferował szybki dostęp do danych. Ponadto, połączenie komplementarnych metod
kodowania nie powoduje utraty zdolności do kodowania inkrementalnego, a nawet ją wspomaga.

% W ramach pracy stworzone zostały dwie składowe metody kodowania: jedna bazująca na dziedziczeniu cech, druga bazująca na
% schematach numerowania. Poprawność opracowanych rozwiązań została udowodniona, a wydajność sprawdzona w środowisku
% testowym. Wyniki obrazują charakterystykę hybrydowej metody indeksowania i dowodzą jej efektywności.

\paragraph{Słowa kluczowe:} indeksowanie relacji przechodnich, hierarchia dziedziczenia wielokrotnego, sprawdzanie
istnienia ścieżki w grafie

\pagenumbering{gobble}
