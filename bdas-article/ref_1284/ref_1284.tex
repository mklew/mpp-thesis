
%%%%%%%%%%%%%%%%%%%%%%% file typeinst.tex %%%%%%%%%%%%%%%%%%%%%%%%%
%
% This is the LaTeX source for the instructions to authors using
% the LaTeX document class 'llncs.cls' for contributions to
% the Lecture Notes in Computer Sciences series.
% http://www.springer.com/lncs       Springer Heidelberg 2006/05/04
%
% It may be used as a template for your own input - copy it
% to a new file with a new name and use it as the basis
% for your article.
%
% NB: the document class 'llncs' has its own and detailed documentation, see
% ftp://ftp.springer.de/data/pubftp/pub/tex/latex/llncs/latex2e/llncsdoc.pdf
%
%%%%%%%%%%%%%%%%%%%%%%%%%%%%%%%%%%%%%%%%%%%%%%%%%%%%%%%%%%%%%%%%%%%


\documentclass[runningheads,a4paper]{llncs}

\usepackage{amssymb}
\setcounter{tocdepth}{3}
\usepackage{graphicx}

\usepackage{url}
\newcommand{\keywords}[1]{\par\addvspace\baselineskip
\noindent\keywordname\enspace\ignorespaces#1}

\begin{document}

\mainmatter

\title{Multi-partition distributed transactions over Cassandra-like database with tunable contention control}

\author{Marek Lewandowski \and Jacek Lewandowski}

\institute{Institute of Computer Science, Warsaw University of Technology, Nowowiejska 15/19, 00-665 Warsaw, Poland\\
\url{http://www.ii.pw.edu.pl}\\
\email{\{marekmlewandowski, lewandowski.jacek\}@gmail.com}}

\maketitle

\begin{abstract}
The amounts of data being processed today are enormous and they require specialized systems 
to store them, access them and do computations. Therefore, a number of NoSql databases and 
big data platforms were built to address this problem. They usually lack of transaction 
support which feature atomicity, consistency, isolation, durability and at the same time 
they are distributed, scalable, and fault tolerant. In this paper we present a novel 
transaction processing framework based on Cassandra storage model. It uses Paxos protocol 
to provide atomicity and consistency of transactions and Cassandra specific read and write 
paths improvements to provide read committed isolation level and durability. Unlike built-in 
Light Weight Transactions (LWT) support in Cassandra, our algorithm can span multiple data 
partitions and provides tunable contention control. We verified correctness and efficiency 
both theoretically and by executing tests over different workloads. The results presented 
in this paper prove the usability and robustness of the designed system. 
\keywords{big data, transactions, cassandra, paxos, consistency, nosql}
\end{abstract}

\section{Introduction}
Big Data is the term used for describing sets of increasing volumes of data, which do not fit 
in a single machine. There are many sources of Big Data, such as system logs, user website 
clicks, financial transactions, weather measurements, data from Internet of Things, and many others.

NoSql databases were created to support storing Big Data and provide the means to analyze it. 
Databases differ in a ways they represent the data, but the key principle remains the same: 
store the data for future analysis. NoSql databases span over hundreds, even thousands servers 
located in different various physical locations and are designed to overcome individual node 
failures and network partitions. They provide guarantees for certain behaviour in face of such
problems which usually means they eventually achieve the consistency of data at some point.

To this end, the ensemble of data is cut into groups of records which are called partitions. 
Partitions are data replication and sharding units, which means that they are distributed across 
the cluster in one or more copies so that the load can be balanced among different server nodes,
and the availability of a distinct chunk of data is increased.

Properties of relational databases, such as ACID compliant transactions are usually sacrificed in
NoSQL-like solutions in favour of high availability, fault tolerance and scalability. Though, 
there exist some solutions which can help to overcome such deficiencies. Unfortunately they come
with either functional or performance limitations which makes them difficult to adopt. 
The functional limitations include constraining a transaction to a single partition of data
or the lack of certain isolation levels. The performance problems are usually related to the 
use of pessimistic locks to achieve a certain level of consistency.

We overcame those problems by modifying Cassandra database. Cassandra is proven to be performant 
and provide some basic transaction support known as LWT. Although the usability of LWT is limited,
the model of storing data makes it a perfect platform for building transaction support extension
which features good performance and rich functionality.

% TODO - jacek

\section{Preliminaries}

% Scalability, 

% Paxos

% Cassandra

% LWT

\section{Multi partition transactions algorithm}

\section{Implementation of the algorithm in Cassandra}

\section{Tests and performance analysis}

\section{Related work}

% TODO - jacek

\section{Summary}

% TODO - jacek

\bibliographystyle{splncs03}
\bibliography{ref_1284} 

\end{document}
