
%%%%%%%%%%%%%%%%%%%%%%% file typeinst.tex %%%%%%%%%%%%%%%%%%%%%%%%%
%
% This is the LaTeX source for the instructions to authors using
% the LaTeX document class 'llncs.cls' for contributions to
% the Lecture Notes in Computer Sciences series.
% http://www.springer.com/lncs       Springer Heidelberg 2006/05/04
%
% It may be used as a template for your own input - copy it
% to a new file with a new name and use it as the basis
% for your article.
%
% NB: the document class 'llncs' has its own and detailed documentation, see
% ftp://ftp.springer.de/data/pubftp/pub/tex/latex/llncs/latex2e/llncsdoc.pdf
%
%%%%%%%%%%%%%%%%%%%%%%%%%%%%%%%%%%%%%%%%%%%%%%%%%%%%%%%%%%%%%%%%%%%


\documentclass[runningheads,a4paper]{llncs}

\usepackage{amssymb}
\setcounter{tocdepth}{3}
\usepackage{graphicx}
\usepackage{xspace}

\usepackage{url}
\usepackage[inline]{enumitem}


\usepackage{mathtools}


% Nowe oznaczenia w Outline
\newcommand{\nodes}{$\mathit{N}$\xspace}
\newcommand{\nodesTx}{$\mathit{N'}$\xspace}
\newcommand{\transaction}{$\Delta$\xspace}
\newcommand{\transactionOne}{$\Delta_1$\xspace}
\newcommand{\txOne}{$\Delta_{1}$\xspace}
\newcommand{\txTwo}{$\Delta_{2}$\xspace}
\newcommand{\transactionj}{$\Delta_{j}$\xspace}
\newcommand{\transactionm}{$\Delta_{m}$\xspace}
\newcommand{\transactioni}[1]{$\Delta_{#1}$\xspace}
\newcommand{\txStateM}{$\Lambda_{new}$\xspace}
\newcommand{\transactions}{$\{\Delta_{i}, \Delta_{j}, ...\}$\xspace}
\newcommand{\txStates}{$\{\Lambda_{i}, \Lambda_{j}, ...\}$\xspace}
\newcommand{\conflictFunction}{$\zeta (\text{\txStateOne, \txStateTwo}) \mapsto ( \mathcal{C}_1, \mathcal{C}_2)$\xspace}

\newcommand{\txLog}{$\mathcal{L}$\xspace}
\newcommand{\conflictingTxSet}{$\mathcal{C}\text{\txStates}$\xspace}
% messages
\newcommand{\beginTransactionMessage}{$\mathit{M}(c, n, \mathit{begin\_transaction}())$\xspace}
\newcommand{\initialTxStateMessage}{$\mathit{M}(n, c, \mathit{initial\_transaction\_state}(\Lambda_{0}))$\xspace}
\newcommand{\selectMessage}{$\mathit{M}(c,n,select(k))$\xspace}
\newcommand{\updateTxStateMessage}{$\mathit{M}(c, n, \mathit{update\_transaction\_state}(\lambda))$\xspace}
\newcommand{\txRollbackMessage}{$\mathit{M}(c,n,\mathit{transaction\_rollback}(\Lambda))$\xspace}
\newcommand{\rollbackMessage}{$\mathit{M}(n_{i}, n_{j}, \mathit{rollback}(\Lambda))$\xspace}
\newcommand{\nodesOfMutations}{$(\tau(k_1) \cup \tau(k_2) \cup ... ) =  \text{\nodesTx}\in\mathit{N}$\xspace}

\newcommand{\txCommitMessage}{$\mathit{M}(c,n, \mathit{transaction\_commit}(\Lambda))$\xspace}
\newcommand{\txCommitResonseMessage}{$\mathit{M}(n_{i},c,\mathit{transaction\_commit\_response}(committed))$\xspace}

\newcommand{\insertMessage}{$\mathit{M}(c, n, \mathit{upsert(\text{\txState}, k,v)})$\xspace}
\newcommand{\NRF}{\mathit{N^{\mathbb{RF}}}}
\newcommand{\database}{$\Omega$\xspace}
\newcommand{\mutation}[2]{$\delta(#1, #2)$\xspace}
\newcommand{\mutations}{$\{\delta_{1}, \delta_{2}, ...\}$\xspace}
\newcommand{\topology}{$\tau$\xspace}

\newcommand{\topologyItem}[2]{$\tau(\text{\txItemi{#1}})) \mapsto \mathit{#2}$}


\newcommand{\paxosRoundId}{$\iota$\xspace}
\newcommand{\paxosRoundIdi}[1]{$\iota_{#1}$\xspace}
\newcommand{\mutationsFull}{$\{\delta_{1}(k_1, v_1), \delta_{2}(k_2, v_2), ...\}$\xspace}
\newcommand{\mutationsFullEnd}{$\{\delta_{1}(k_1, v_1), \delta_{2}(k_2, v_2), ..., \delta_{i}(k_i, v_i)\}$\xspace}

\newcommand{\transactionFull}{$\Delta(\delta_{1}, \delta_{2}, ...)$\xspace}



\newcommand{\txItem}{$\lambda$\xspace}
\newcommand{\txItems}{$\{\lambda_{1}, \lambda_{2}, ...\}$\xspace}
\newcommand{\txItemi}[1]{$\lambda_{#1}$\xspace}
\newcommand{\txState}{$\Lambda$\xspace}
\newcommand{\txStatei}[1]{$\Lambda_{#1}$\xspace}
\newcommand{\txStateOne}{$\Lambda_1$\xspace}
\newcommand{\txStateTwo}{$\Lambda_2$\xspace}
\newcommand{\txStateCommitted}{$\Lambda_{learnt}$\xspace}

\newcommand{\txIndex}{$\chi$\xspace}
\newcommand{\txStorage}{$\omega$\xspace}
% Shortcuts
\newcommand{\paxos}{\emph{Paxos}\xspace}
\newcommand{\mpt}{\emph{MPT}\xspace}
\newcommand{\lwt}{\emph{LWT}\xspace}
\newcommand{\RF}[1]{\emph{$\mathbb{RF}=#1$}\xspace}
\newcommand{\RFalone}{$\mathbb{RF}$\xspace}
\newcommand{\RFaloneInMath}{\mathbb{RF}}
\newcommand{\RFalonePrim}{$\mathbb{RF'}$\xspace}

\newcommand{\ballot}{$\beta$\xspace}
\newcommand{\coordinator}{$C$\xspace}
\newcommand{\paxosValue}{$v$\xspace}

\newcommand{\client}{$c$\xspace}


\newcommand{\tx}[1]{$t_{#1}$\xspace}

\newcommand{\node}[1]{$n_{#1}$\xspace}

\newcommand{\N}[1]{$N=#1$\xspace}
\newcommand{\mptrequest}[2]{$r_{#1\mapsto#2}$\xspace}

\newcommand{\clientReq}[1]{$req_{c\mapsto n_{#1}}$\xspace}
\newcommand{\nodeReqResponse}[1]{$resp_{n_{#1}\mapsto c}$\xspace}
\newcommand{\nodeMessage}[2]{$m_{n_{#1} \mapsto n_{#2}}$\xspace}

\newcommand{\key}{$k$\xspace}
\newcommand{\keyi}[1]{$k_{#1}$\xspace}
\newcommand{\kvalue}{$v$\xspace}
\newcommand{\kvaluei}[1]{$v_{#1}$\xspace}
\newcommand{\kv}{$(k, v)$\xspace}
\newcommand{\kvi}[2]{$(k_{#1}, v_{#2})$\xspace}
\newcommand{\mutationsi}[1]{$\Delta(t_{#1})$\xspace}

%\newcommand{\topology}[1]{$\tau(T, H, k) \mapsto \{ \text{#1} \}$}
\newcommand{\topologyTk}[1]{$\tau(T, tk) \mapsto \{\text{ #1 }\}$}

\newcommand{\keywords}[1]{\par\addvspace\baselineskip
\noindent\keywordname\enspace\ignorespaces#1}

\begin{document}

\mainmatter

\title{Multi-partition distributed transactions over Cassandra-like database with tunable contention control}

\author{Marek Lewandowski \and Jacek Lewandowski}

\institute{Institute of Computer Science, Warsaw University of Technology, Nowowiejska 15/19, 00-665 Warsaw, Poland\\
\url{http://www.ii.pw.edu.pl}\\
\email{\{marek.m.lewandowski, lewandowski.jacek\}@gmail.com}}

\maketitle

\begin{abstract} The amounts of data being processed today are enormous and they
require specialized systems  to store them, access them and do computations.
Therefore, a number of NoSql databases and  big data platforms were built to
address this problem. They usually lack transaction support which features
atomicity, consistency, isolation, durability and at the same time  they are
distributed, scalable, and fault tolerant. In this paper, we present a novel
transaction processing framework based on Cassandra storage model. It uses Paxos
protocol  to provide atomicity and consistency of transactions and Cassandra
specific read and write  paths improvements to provide read committed isolation
level and durability. Unlike built-in  Light Weight Transactions (LWT) support
in Cassandra, our algorithm can span multiple data  partitions and provides
tunable contention control. We verified correctness and efficiency  both
theoretically and by executing tests over different workloads. The results
presented  in this paper prove the usability and robustness of the designed
system.  \keywords{big data, transactions, cassandra, paxos, consistency, nosql}
\end{abstract}

\section{Introduction} Big Data is the term used for describing sets of
increasing volumes of data, which do not fit  in a single machine. There are
many sources of Big Data, such as system logs, user website  clicks, financial
transactions, weather measurements, data from Internet of Things, and many
others.

NoSql databases were created to support storing Big Data and provide the means
to analyze it.  Databases differ in ways they represent the data, but the key
principle remains the same:  store the data for future analysis. NoSql databases
span over hundreds, even thousands servers  located in various physical
locations and are designed to overcome individual node  failures and network
partitions. They provide guarantees for certain behaviour in face of such
problems which usually means they eventually achieve the consistency of data at
some point.

To this end, the ensemble of data is cut into groups of records which are called
partitions.  Partitions are data replication and sharding units, which means
that they are distributed across  the cluster in one or more copies so that the
load can be balanced among different server nodes, and the availability of a
distinct chunk of data is increased.

Properties of relational databases, such as ACID compliant transactions are
usually sacrificed in NoSQL-like solutions in favour of high availability, fault
tolerance and scalability. Though,  there exist some solutions which can help to
overcome such deficiencies. Unfortunately, they come with either functional or
performance limitations which make them difficult to adopt.  The functional
limitations include constraining a transaction to a single partition of data or
the lack of certain isolation levels. The performance problems are usually
related to the  use of pessimistic locks to achieve a certain level of
consistency.

We overcame those problems by modifying Cassandra database. Cassandra is proven
to be performant  and provide some basic transaction support known as LWT.
Although the usability of LWT is limited, the model of storing data makes it a
perfect platform for building transaction support extension which features good
performance and rich functionality.

\section{Preliminaries}

Database $\Omega$ consists of key-value pairs $\{(k_{i},v_{j}), (k_{k},v_{m}),...\}, k_i\in\mathit{K}, v_i\in\mathit{V}$ stored by cluster of nodes $\{n_1, n_2, ...\}, n_i\in\mathit{N}$.
Changes to data are represented by \emph{mutations} $\delta(k,v)$. 
Each mutation is stored in nodes identified by topology $\tau$ (Definition \ref{def:topology}).
Transaction is a set of mutations $\Delta(\delta_{1}, \delta_{2}, ...)$, where mutation \mutation{k}{v} is an operation
on key $k$ with value $v$.

\begin{definition}
  \label{def:topology}
  \emph{Topology} is a partitioning function $\tau:\mathit{K} \mapsto \mathit{N}^{\mathbb{RF}}$ which for a given $k$ returns a set of replica nodes of size $\mathbb{RF}$. 
\end{definition}

Client $c_{i}\in\mathit{C}$ is an actor, which performs transactions $\Delta_{1}, \Delta_{2}, ...$ and communicates to a cluster \nodes using messages $\mathit{M}$. Client \client begins a transaction \transaction by sending a message \beginTransactionMessage to any node $n\in\mathit{N}$ followed by receiving a message \initialTxStateMessage, which includes initial \emph{transaction state} (Definition \ref{def:transactionStateChapter4}), which will reflect all subsequent changes done in \transaction.

During the transaction \transaction, client \client sends messages of two types: mutations and selects \selectMessage. All mutations within a transaction \transaction are isolated from other transactions - they are private to the enclosing transaction until it is committed. For the duration of the transaction, mutations are stored in a \emph{private transaction storage}, which is a local data structure at each \node{i}. Select operations query against current state of the database, thus transactions provide read-committed isolation level. This means that transactions always get most recent committed state of data, thus a subsequent query for the same key can yield different value if there was a successfully committed transaction in between select queries. 

After each mutation client \client receives a message \updateTxStateMessage, which includes a transaction item \txItem referencing that mutation. A client is responsible for tracking changes performed in a transaction, thus it has to append all the received transaction items to the initial transaction state \txStatei{0}. 

Transactions are decoupled from physical network connection and so is transaction state which means that if a client loses connection to a cluster due to some kind of network failure then transaction can be completed after the client connects back to the cluster.

\begin{definition}
  \label{def:transactionItemChapter4}
  \emph{Transaction item} $\lambda(\mathit{keyspace}, \mathit{table}, \mathit{token})$ 
  is a reference to a single mutation \mutation{k}{v} of transaction \transaction, which indicates that replica nodes of $k$ store \mutation{k}{v} in \emph{private transaction storage}.  
  It is a triple with \begin{enumerate*}[label=\alph*)]
    \item $\mathit{keyspace}$ -- name of a keyspace,
    \item $\mathit{table}$ -- name of a table,
    \item $\mathit{token}$ -- value of a token, which is a value computed by hash function from Cassandra key $k_p$, which is used by topology \topology to determine replicas $\NRF_k$. Cassandra key $k(k_p,k_c)$ consists of two keys: \begin{enumerate*}[label=\alph*)]
    \item $k_p$ is a partitioning key, which is used by topology \topology to determine $\mathit{N}^{RF}_k$,
    \item $k_c$ is a clustering key, which determines column in wide rows.
  \end{enumerate*}
  \end{enumerate*}
\end{definition}

\begin{definition}
  \label{def:transactionStateChapter4}
  \emph{Transaction state} $\Lambda(\mathit{id}, \{\lambda_{i}, \lambda_{j}, ...\})$ reflects changes done in transaction \transaction. Transaction state allows to identify nodes $N' \subset N$, which are affected by \transaction and store its \mutations. It is a pair of unique id of type \emph{timeuuid}, which is UUID with encoded timestamp \cite{CassandraUUID} and a set of transaction items (Definition \ref{def:transactionItemChapter4}).
\end{definition}

% When client \client finishes its planned operations transaction \transaction can be either committed or rolled back.
To perform rollback client sends \txRollbackMessage, and then $n$ sends out rollback messages to other nodes \rollbackMessage, where nodes $\{n_j,...,n_k\}$ are identified by \txItems, which makes them remove private storage for that transaction and mark \transaction as rolled back in an another local data structure called \emph{transaction log}.

In order to commit transaction \transaction a client \client sends message \txCommitMessage and subsequently \node{i} becomes the proposer of the transaction and is responsible for orchestrating distributed consensus round among nodes affected by transaction \transaction. %referenced by transaction items.%
When consensus is reached each node moves private data of transaction \transaction from private transaction storage to the main storage and marks \transaction, as committed in the transaction log. The client \client receives message \txCommitResonseMessage about successfully committed transaction.

\section{Multi partition transactions algorithm}

There can be many concurrent transactions and some of them mutate values associated with the same keys, thus concurrency control guaranteeing that interfering transactions are not committed at the same time is required. Commit procedure solves this problem by grouping such transactions and performing distributed consensus round, which selects a single transaction that is committed and rollbacks the others. Commit procedure uses two local data structures \emph{transaction index}, which is responsible for identifying interfering transactions and grouping them into the same consensus rounds and \emph{transaction log}, which records committed or rolled back transactions.

\subsection{Representation of a transaction}

Transaction state (Definition \ref{def:transactionStateChapter4}) is used to identify transaction \transaction and to track mutations done in \transaction, where each mutation is represented by \emph{transaction item} \txItem (Definition \ref{def:transactionItemChapter4}) received in a response to a mutation message. A client \client begins transaction by sending \beginTransactionMessage and receives empty transaction state \txState in a message  \initialTxStateMessage. Values of \mutationsFull tracked by transaction items \txItems are unknown to \txItems, but each value $v$ can be found at replica nodes $N^{\mathbb{RF}}(n_{i},n{j},...)\subset \mathit{N}$ by topology function \topology applied to each \txItem.

\subsubsection{Tracking updates}

Transaction state \txState changes as transaction \transaction progresses and its set of transaction items have to be updated with new transaction items after each mutation message. 

% \begin{algorithm}
%   \caption{Updating transaction state after two mutations}
%   \label{alg:updateTxState}
%   \begin{algorithmic}       
%     \State \beginTransactionMessage
%     \State \initialTxStateMessage
%     \State $\mathit{M}(c, n_{i}, \mathit{upsert(k_{1},v_{1})})$
%     \State $\mathit{M}(n_{i}, c, \mathit{update\_transaction\_state}(\lambda_{1}))$
%     \State $\Lambda() \gets \Lambda() \oplus \lambda_{1}$
%     \State $\mathit{M}(c, n_{i}, \mathit{upsert(k_{2},v_{2})})$
%     \State $\mathit{M}(n_{i}, c, \mathit{update\_transaction\_state}(\lambda_{2}))$
%     \State $\Lambda(\lambda_{1}) \gets \Lambda(\lambda_{1}) \oplus \lambda_{2}$
%     \State $\mathit{M}(c, n_{i}, \mathit{transaction\_commit}(\Lambda(\lambda_{1}, \lambda_{2})))$
%     \State $\mathit{M}(n_{i}, c, \mathit{transaction\_commit\_response}(committed))$    
%   \end{algorithmic}
% \end{algorithm}

A client is responsible for keeping track of all performed operations within a transaction. Transaction state is assumed to be a source of knowledge about a transaction, which means that it has valid information about which replicas are affected by a transaction. Note that, client i.e. database driver has to keep track of transaction items and append them to the transaction state after each transactional mutation, but the end user of the driver is potentially unaware of transaction items.

\subsection{Isolation from other transactions}
Transactions must run in isolation from each other and provide read-committed isolation level, thus transactions cannot see each other, nor anyone else can see effects of transactions before it completes with commit or rollback. Expected read-committed isolation is different than the one known from RDBMS, because in RDBMS such isolation holds write-locks for the duration of a transaction, therefore if a transaction writes then subsequent reads return written values. We want to avoid distributed locks, which decrease performance and cause deadlocks, therefore read-committed isolation in \mpt algorithm does not hold any locks, thus each \selectMessage receives current $v$, as it is in \database.

Each mutation \mutation{k}{v} done in transaction \transaction should be stored on replicas responsible for given key keeping it private for that \transaction. Each mutation should be isolated from transactions other than \transaction and accessible only by \transaction. In order to satisfy isolation requirement, \emph{private transaction storage} has to be present on each node. 

Each mutation message is executed by \node{i}, but instead of operating on current live data of \database, mutations \mutations are put into private storage presented in Definition \ref{def:privateTransactionStorage}. 

\begin{definition}
  \label{def:privateTransactionStorage}
  \emph{Private transaction storage} \txStorage is a local data structure on each node \node{i} which stores mutations \mutations of transaction \transaction in isolation from other transactions \transactions and live data set of database \database and provides operations listed below: 
  \begin{itemize}
    \item $\oplus(\text{\txState}, \text{\mutation{k}{v}}) \mapsto \text{\txItemi{k}}$ -- stores mutation \mutation{k}{v} of transaction \transactionj represented by transaction state \txState and returns transaction item \txItemi{k}. 
    \item $\mathit{clear}(\text{\txState})$ -- removes all mutations \mutations associated with transaction \transactionj represented by transaction state \txState on node \node{i}
    \item $\mathit{get}(\text{\txState}) \mapsto \text{\mutations}$ -- returns set of mutations of transaction \transactionj represented by transaction state \txState stored on replica \node{i} 
  \end{itemize}
\end{definition}

Definition \ref{def:privateTransactionStorage} abstracts from memory and time, but an implementation of private transaction storage has to $\mathit{clear}(\text{\txState})$ after certain timeout in order to free memory (assuming in-memory storage) and to be resilient to failures of clients, which can fail and abandon transaction \transactionj at any point in time. From a perspective of the algorithm, it does not matter whether mutations \mutations are stored in memory or in any other way as long as \mutations are accessible during transaction's execution. 

\paragraph{Quorum requirements}

Each mutation \mutation{k}{v} of transaction \transactionj should be written to at least quorum of replicas $N^{\mathbb{RF}} = \tau(k)$ (see Definition \ref{def:topology}) in order to depend on majority during the commit process. If quorum is not available, then \transactionj needs to be rolled back. 

\subsection{Consensus by Paxos}

We use \paxos to reach consensus among nodes \nodesTx, which select a single transaction \transaction to commit out of set of concurrent \emph{conflicting} transactions \transactions (Definition \ref{def:conflictingTransactionsSet}). \paxos provides consensus on a single value among many values proposed during the same \paxos round, therefore in order to commit a single transaction \transaction, all conflicting \transactions must participate in the same \paxos round, which is not trivial considering many mutations \mutationsFull, as there is no longer single key $k$, which would identify \paxos round, as it is in \lwt.

\begin{definition}
  \label{def:conflictingTransactionsSet}
  \emph{Conflicting transactions set} - denoted by \conflictingTxSet is a set of transactions where all \transactions include at least single common mutation $\delta(k)$, which mutates value associated with the same key $k$.
\end{definition}

\subsubsection{Proposed transaction state}

Our true value is transaction \transactionFull, however mutations \mutationsFull are distributed over nodes
\nodesOfMutations. As a result of that it is impossible to propose \transaction itself, but it is possible to propose transaction state \txState, which references mutations \mutations, thus proposed \paxos \emph{value} in \mpt is \txState. Any node \node{i} can try to commit \transaction given \txState, because \node{i} can check which nodes participate in \transaction and where are private data stored.

\subsubsection{Reaching the same Paxos round}

Conflicting transactions \conflictingTxSet must participate in the same \paxos round identified by \paxos round id \paxosRoundId. If there are many transactions \transactions being committed at the same time and those \transactions are in conflict with each other, then only single transaction \transaction should get committed and rest of them should be rolled back. Nodes \nodesTx must agree on \paxos value, which is transaction state \txState, thus agree on which \transaction is committed. Rest of transactions $(\mathcal{C}\text{\txStates} - \text{\txState})$ can be rolled back, as long as they participate in the same \paxos round \paxosRoundId. In case of \lwt \paxosRoundId is determined by \emph{k}, however \mpt supports more than one key, therefore we need a function which maps \txState $\mapsto $ \paxosRoundId with properties: 
\begin{enumerate*}[label=\alph*)]
  \item it maps conflicting transactions to the same \paxosRoundId,
  \item it maps non-conflicting transactions to different $\iota'$.
\end{enumerate*}

\subsubsection{Conflict function}

Function \txState $\mapsto $ \paxosRoundId can be a composition of two functions: \mbox{\txState $\mapsto $ \conflictingTxSet} and \mbox{\conflictingTxSet $\mapsto$ \paxosRoundId}, where the former groups transactions into conflicting sets and the latter assigns \paxos round id \paxosRoundId to each set. In order to group conflicting transactions we need a function, which compares transaction states \txStateOne and \txStateTwo in pairs and detects whether \txStateOne and \txStateTwo are in conflict. Definition \ref{def:conflictFunction} presents such function.

\begin{definition}
  \label{def:conflictFunction}
  \emph{Conflict function} denoted, as $\zeta (\text{\txStateOne, \txStateTwo}) \mapsto ( \mathcal{C}_1, \mathcal{C}_2)$, where $\mathcal{C}_1 = \mathcal{C}(\text{\txStateOne, \txStateTwo}) \wedge \mathcal{C}_2 = \emptyset $ or $\mathcal{C}_1 = \mathcal{C}(\text{\txStateOne}) \wedge \mathcal{C}_2=\mathcal{C}(\text{\txStateTwo})$, the former case is when transactions are in conflict and contain at least single $\delta$ for the same $k$, the latter otherwise.
\end{definition}
 
\subsection{Transaction index}

Transaction index denoted, as \txIndex is a service, local to a node, which provides function \mbox{\txState $\mapsto $ \paxosRoundId} required to reach the same \paxos round for conflicting transactions.

Registration of transaction state \txState in \txIndex on each \node{i} $\in$\nodesTx is precondition to the commit procedure of \mpt. Since \txIndex is local, each transaction \transaction receives different \paxos round id \paxosRoundId on different nodes, but the value of \paxosRoundId is never used outside of a node, thus \paxosRoundId does not have to be globally the same. \paxosRoundId is used only to identify \paxos round for \conflictingTxSet at each \node{i}. If \transaction participates in the same \paxos round identified by \paxosRoundId as other conflicting transactions, then only single \transaction is committed after \nodesTx agree on the proposed \txState.

Transaction index stores sets of conflicting transactions \conflictingTxSet and assigns \paxosRoundId to each set. When a new transaction state \txStateM is registered, \txIndex tries to find a set \conflictingTxSet to which \txStateM can be added by applying conflict function \conflictFunction to \txStateM and each transaction \transaction in each set. If there is a conflict, sets can be merged together, thus \txStateM joins \conflictingTxSet and receives the same \paxosRoundId. Registration algorithm is depicted in Algorithm \ref{alg:indexRegistration}.

During the commit procedure of \mpt transaction \transactionFull has to register in each quorum for each replica subset $(\NRF_{k_1} \cup \NRF_{k_2} \cup ... ) = (\tau(k_1) \cup \tau(k_2) \cup ... ) = \text{\nodesTx}\subset\mathit{N}$. Moreover transaction \transaction cannot be registered in transaction index \txIndex when it can obtain more than one \paxos round id \paxosRoundId. These two requirements guarantee that if transaction \txOne and transaction \txTwo are conflicting for some key $k$, then there exists node \node{i} $\in N^{\RFaloneInMath}_{k}$ on which \txOne and \txTwo receive the same \paxosRoundId, thus both transactions are part of the same \paxos round and eventually only one is committed.

\subsubsection{Example of registration}

Assume we have transaction states 
\txStatei{1}$(\text{\txItemi{1}},\text{\txItemi{2}}, \text{\txItemi{3}})$,
\txStatei{2}$(\text{\txItemi{2}},\text{\txItemi{3}})$, \RF{3} and $N=5$. 
Note that both transactions have transaction item \txItemi{2}, thus mutate the same key. Given that \txItem is equivalent of a key (Definition \ref{def:transactionItemChapter3}) we can use topology \topology to find replicas on which transactions have to register. Assume topology \topologyItem{1}{N'} and \topologyItem{2}{N''} where $N'=(n_1, n_2, n_3)$ and $N''=(n_3,n_4,n_5)$.

Transaction state \txStatei{1} has to register in quorum of $N'$ and quorum of $N''$; \txStatei{2} has to register in quorum of $N''$. $quorum(N'')$ is equal to one of $((n_3,n_4),(n_3,n_5),(n_4,n_5))$, therefore $\mathit{quorum(N''}) \cap \mathit{quorum(N''}) \neq \emptyset$, thus exists a $n\in N''$ on which boths transactions are registered. \txStatei{1} and \txStatei{2} receive the same \paxosRoundId on $n$, thus if one is committed, the other has to be rolled back.

\subsubsection{Rolling back concurrent transactions}
Transaction index knows which \paxos round id \paxosRoundId to assign to which transaction state \txState and it knows conflicting transactions set \conflictingTxSet to which \txState is assigned. This knowledge is used to rollback conflicting transactions when \paxos round \paxosRoundId is finished at node \node{i} and \txStateCommitted is learnt by \node{i}. The node calls $\mathit{clear}(\text{\txState})$ function of private transaction storage for each \mbox{\txState $\in$ $(\text{\conflictingTxSet} - \text{\txStateCommitted})$}.

If a client \client wants to rollback transaction \transaction then that client sends a message \txRollbackMessage after which \node{i} identifies \nodesTx and broadcasts message \rollbackMessage for each \node{j} $\in$ \nodesTx. When \node{j} receives such message it uses $\mathit{clear}(\text{\txState})$ function of private transaction storage. Transaction \transaction is rolled back after all \mbox{$n\in \text{\nodesTx}$} clear private data.

\subsection{Transaction Log}
\label{sec:mpp:transactionLog}

\emph{Transaction Log} \txLog is a local data structure present at each \node{i}$\in\text{\nodes}$. Its responsibility is to record committed and rolled back transactions. It provides the following operations:
  \begin{enumerate*}[label=\alph*)]
    \item $\mathit{record\_as\_committed(\text{\txState})}$ -- record a committed transaction
    \item $\mathit{record\_as\_rolled\_back(\text{\txState})}$ -- record a rolled back transaction
    \item $\mathit{find(\text{\txState})} \mapsto \mathit{log\_state}$ -- finds information concerning a transaction. Returns $\mathit{log\_state}$ which can be one of three: \emph{committed}, \emph{rolled back} or \emph{unknown} when transaction is not recorded in the log.
  \end{enumerate*} 

\section{Implementation of the algorithm in Cassandra}

\section{Tests and performance analysis}

\section{Related work}

There are a lot of NoSQL database - a web page which collects the information about the known systems of this kind \cite{NoSqlDatabasesOrg} reports over 225 existing NoSql databases. Although they are called NoSQL, not all of them fall into this category because of horizontal scalability and high availability support. As mentioned in the introduction, we can find implementations which support transactions to some degree. They include MarkLogic, IBM Informix, Datomic, Oracle and others. They use various techniques to achieve ACID properties, which we will cover below. 

\subsection{2 and 3 phase commit}
\subsubsection{2 phase commit}
2 phase commit operates in two distinct phases \cite{Bernstein:1987}: 
\begin{enumerate} 
	\item Contact every node, propose a value and gather responses 
	\item If all nodes agree, contact every node again to let it know about an
	agreement. Otherwise, contact every node to abort the consensus. 
\end{enumerate}

A value is proposed by any node, which is then called the \emph{coordinator} who initiates a round of 2PC. Note that, in 2PC nodes cannot suggest other value than the proposed one, they can only accept it, or decline it. Another proposal can be done only through another round. 

Fault tolerance is not provided by 2PC. The 2PC solves the consensus problem in the absence of failures in 3 rounds. The first  round of messages for a proposal, the second round for votes and the third round for commit or abort. In this case there are 3n messages send for the whole protocol lifecycle. In the presence of failures,  the protocol may take additional 2 rounds to recover. Due to blocking nature of the protocol, timeouts need to be employed to terminate properly.

\subsubsection{3 phase commit}

3 phase commit splits the second phase of 2PC into two sub-phases: prepare to commit phase and commit phase, thus it has three phases in total \cite{Bernstein:1987}.

It requires 5 rounds and 5n messages in the absensce of failures to reach a consensus.  Unlike 2PC, it is not blocking. 3PC can accept a configurable number of f failures, because the coordinator can move to phase 3 after it receives f + 1 acknowledges of prepare to commit. 3PC does not work well during the network partition, because if part of nodes on one side of the partition received a prepare to commit message and all nodes on the other side did not receive that message, then recovery node on the former side commits a transaction, whereas the other recovery node aborts it. After network partition is gone, nodes remain in the inconsistent state, in which transaction is committed and aborted at the same time.

\subsection{Multiversioning} 

Multi version concurrency control \cite{Bernstein:1983} is known in the area of distributed databases as well as in the area of software transactional memory \cite{Perelman:2010}. It can outperform 2-phase commit in terms of efficiency and it handles failures in a more efficient way \cite{Halici:1991}. The main idea behind this approach is to attach a version to each operation so that both reads and writes are applied on an exact version of data. This can significantly increase concurrency of transactions by reducing the number of conflicts. However, as noted in \cite{Faleiro:2015}, this approach make performing serializable transactions difficult.

\subsection{RAMP transactions} 
\emph{Read Atomic Multi Partition} \cite{Bailis:2014} was designed to address the requirements of the following use cases: secondary indexing, foreign key enforcement, and materialized view maintenance. \emph{RAMP} identifies new transaction isolation level \emph{Read Atomic}, which assures consistent visibility of changes made by the transaction: either all or none of each transaction’s updates are seen by other transactions.

The key principle behind \emph{RAMP} is its ability to detect non-atomic partial read, and to repair it with additional rounds of communication with the nodes by the means of the metadata attached to each write and multi-versioning, which means that data are accessible in different versions and each modification to data creates a new version. Versions are either committed or not yet committed, in which case the data can be discarded by aborting the transaction [21, p. 6]. There are three variants of \emph{RAMP} algorithm which differ in the type of the metadata. RAMP is linearly scalable, which was tested during trials with the number of nodes spanning up to 100 servers \cite[p. 10]{Bailis:2014}.

Fault tolerance \emph{RAMP} is resilient to failures of part of the nodes in the cluster – for the reason that \emph{RAMP}’s transaction does not contact nodes that are not affected by the transaction. The number of rounds varies between 1 and 2 depending on the existence of the race and algorithm variant. \emph{RAMP} does not block, read transactions race write transactions, but such races are detected and handled by \emph{RAMP}.

\subsection{Consensus algorithms}

\subsubsection{Paxos}

\paxos is an algorithm to solve distributed consensus problem \cite{Lamport1998partTimeParliment} \cite{lamport2001paxosMadeSimple}, which was used, as the solution to the distributed consensus problem as part of the larger systems in \cite{chandra2007PaxosMadeLive}, \cite{lampson1996build}, and more recently in the Google's Chubby service \cite{burrows2006chubby}, which is a lock service intended to provide coarse-grained locking, as well as reliable storage. \paxos is resilient to $(\frac{n}{2}-1)$ failures of nodes. It implements optimistic concurrency control thus it is nonblocking which means that a transaction can be interrupted when a conflict occurs. The message complexity is similar to 3PC.

\subsubsection{Raft}

The aim of \emph{Raft} designers was to create a consensus algorithm which would be more understandable to its predecessor \cite{ongaro2014search}. The main idea behind the algorithm is decomposition of the consensus problem into distinct phases \begin{enumerate*}[label=\alph*)]
  \item leader election,
  \item log replication,
  \item membership changes.
\end{enumerate*} 
\emph{Raft} is fault tolerant, because the leader has to acquire only majority of acknowledges, thus is resilient to failure of $(\frac{n}{2}-1)$ nodes, same like \paxos. However, failure of the leader causes delay because a new leader has to be elected before client's request is handled. In principle \emph{Raft} does not block, because all client requests are handled by the same leader. However, election of the leader can be delayed in case of \emph{split vote}, in which two candidates receive the same number of votes, thus neither of them obtained majority vote and an election has to start over in the next term after a certain timeout. Approximating message complexity of \emph{Raft} is not that clear: assuming that the leader is already elected, a new value is learnt by followers in at least $3n$ messages. However, election process is an additional overhead. Moreover, heartbeat messages are exchanged, thus the total number of messages exchanged can be greater than in previous algorithms.

\section{Summary}

% TODO - jacek

\bibliographystyle{splncs03}
\bibliography{ref_1284} 

\end{document}
