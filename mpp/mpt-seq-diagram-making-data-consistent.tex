%!TEX root = ../thesis.tex
\newcommand{\setupMessage}{$\mathit{M}(n_{i},n_{j}, \mathit{setup\_transition}(\Lambda))$\xspace}
\newcommand{\setupResponseMessage}{$\mathit{M}(n_{j},n_{i}, \mathit{setup\_transition\_result}(success, log))$\xspace}
\afterpage{
	\thispagestyle{empty}
	\begin{figure}[H]
  \centering
	\subfloat[The commit procedure]{%
      \setlength{\unitlength}{0.8cm}
      \begin{sequencediagram} 
  	% \newthread[green!30	]{c}{\client} 
  	\newthread{ni}{\node{i}}
  	\tikzstyle{inststyle}+=[bottom color=blue!30,
		top color=blue!30, rounded corners=3mm]
  	\newinst[2]{nj}{\node{j}}
  	\newinst{nk}{\node{k}}
  	\newinst{nl}{\node{l}}

	\mess{ni}{ setup }{nj}

	\begin{sdblock}{setup transition}{}
		\begin{callself}{nj}{data consistency check}{} 
		\postlevel
		\begin{call}{nj}{query items()}{nk}{mutations}
			\postlevel
		\end{call}
		\postlevel
		\begin{call}{nj}{query items()}{nl}{mutations}
			\postlevel
		\end{call}
		\end{callself}
		\postlevel
		\begin{callself}{nj}{register in \txIndex}{} 
			\postlevel
		\end{callself}
		
	\end{sdblock}
	\mess{nj}{setup response}{ni}
	
	\end{sequencediagram}
    } \par
    \subfloat[Message labels]{      
      \begin{tabular}{|p{3cm}|p{9cm}|}
        \toprule
        label & description \\ \midrule
        query items() & if \node{j} does not have whole subset of mutations \mutations in its private transaction storage \txStorage then \node{j} sends messages to other replica nodes, which reply with mutations stored in their \txStorage \\
        setup & \node{i} sends message \setupMessage \\ 
        transition() & \node{i} sends transition messages to $n\in \text{\nodesTx}$ \\ 
        setup response & \node{j} responds with setup results in message \setupResponseMessage \\ \bottomrule
      \end{tabular}
    }
  \caption{The setup transition}
  \label{fig:seqSetupTransition}
\end{figure}
}
