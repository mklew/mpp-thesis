%!TEX root = ../thesis.tex

\section{Representation of transaction}
Transaction state has \emph{unique id} (UUID\footnote{universally unique identifier}) along with all operations and data done within transaction, thus all modifications. 
Each modification is performed for one key. Key belongs to a partition, thus modification is performed for single partition. There can be many modifications done within transaction. 

\subsection{Tokens}
Each key has a token which belongs to the token ring. Token for key is computed by hash function. \mpp is not dependent on concrete hash function. Tokens are numbers. In case of \emph{JVM} tokens are of type \emph{long} which occupies 8 bytes of memory. Keys are not restricted by any type, thus key can be a 100 character long string which requires around $232$ bytes\footnote{concrete size depends on \emph{JVM} options used}. Size of the key is then 29 times size of a token. Complex keys can have even bigger size. Token is deterministic in size and in most cases occupies less memory than key itself.

Ranges of token ring are divided among nodes in cluster. Division and assignment of ranges to nodes is a topology. Given topology replicas of particular key can be identified. Concretely, token is computed from key and then topology tells us which nodes are assigned to a token range that includes the token. As a consequence, any node in the cluster can point to replicas of particular key or its token.

%Given token and topology of cluster it is known which nodes are replicas responsible for that token. This is how such databases operate. Each node knows that tokens in range (1, 100) are in nodes 1,2,3 and tokens in range (101, 200) in nodes 4,5,6 and so on. Note that this is trivial example.
%Therefore node knows which nodes to ask for data with certain key because of token computed from that key and topology of cluster.

\subsection{Transaction state}
Transaction state has to allow to find all modifications done in transaction. Transaction state is updated with each operation done within transaction. Each update is a \emph{transaction item}. \emph{Transaction item} is a pointer to the operation, thus to its data. Since there are many transactions, transaction state should use least amount of memory possible. Tokens can be used to keep memory usage in bounds. Data structure is proposed at listing (\ref{lst:txState}).
%Coming back to transaction state, it needs to have information about what pieces of data were modified. Requirements recap:

% TODO problem z label
\begin{lstlisting}[language=Java,style=outcode,label={lst:txState},caption={Transaction State data structure}]
class TransactionState
{
    UUID id
    Collection<TransactionItem> transactionItems    
}

class TransactionItem
{
    String tableName
    Token token
}
\end{lstlisting}


% has a \emph{table name} field which tells us
%* Transaction state has to allow to find all modifications
%* Transaction state might be passed around many times between nodes, therefore it has to lightweight.
%* Transaction state has to uniquely identify transaction



% TODO nie mozna uzywac 'something', 'so', 'thing'
%Transaction state is unique id plus something that allows us to find all modifications. That something is called transaction item, which has token and table name. 

Transaction items point to concrete replicas. Existance of transaction item means that replicas identified by \emph{token} store modification of a key done in transaction. Keys are unknown to transaction state, but keys are known to replicas. Each replica has full information about key and operations performed on its value. 

Transaction item has table name, because two tables can have same keys which would result in same tokens and it is necessary to differentiate between those two keys, hence table name is added. This can be extended to any other name spacing. Note that, token ring is unaware of database schema, therefore token alone only directs us to nodes with modified data, but to know what has been modified, we also need names of tables.

%* Tokens in each transaction item point to concrete replicas which store modification done in transaction
 
% Transaction State is just a container for all of these transaction items.

Transaction state is concise data structure, because it uses tokens instead of keys. 

\subsubsection{Without tokens}
Transaction state with keys instead of tokens has same functionality, but varies in memory usage. This is why \mpp is applicable to \emph{key-value} databases in general.
%Same functionality could be achieved storing real keys in transaction items, but then each key could have arbitrary size and same would apply to transaction state.

\subsubsection{Without transaction items}
Transaction items identify nodes that own part of transaction's data. 
In case there were are no transaction items, all we would know would be an id of a transaction.
Using just an id we would need to ask each node in the cluster if it has data associated with transaction. As a corollary algorithm would be dependent on all nodes in the cluster, thus resilience to partial failure would be lost.

\subsubsection{Updates}
Transaction state changes as transaction progresses. It has to be updated with new transaction items for each operation. As a corollary, each operation should return new transaction item which can be appended to transaction state. 


% algorytm z pakietu algorithm2e
% \begin{algorithm}
% \SetKwFunction{startTx}{Begin Transaction}
% \SetKwFunction{updateOperation}{Update Operation}
% \SetKwFunction{commitTx}{CommitTransaction}
% \DontPrintSemicolon
% \KwData{$k$ key}
% \KwData{$value$ of $k$ to update in transaction}
% %\KwResult{$G’=(X,V)$ with $V\subseteq U$ such that  is an interval order.}
% \Begin{
%   $transactionState \longleftarrow BeginTransaction()$ \;
%   $transactionState \longleftarrow \startTx()$ \;
%   $id \longleftarrow transactionState.id$ \;
%   no comment here\;
%   $transactionItem \longleftarrow \updateOperation(id, k, value)$\;
%   $transactionState.addItem(transactionItem)$
%   $result \longleftarrow \commitTx(transactionState)$  
%   }
% \caption{Updating transaction state after an operation222\label{IR}}
% \end{algorithm}

% algorytm z pakietu algorithm

% \begin{algorithmic}[1]
% \Begin
% \State $sum:=0$;
% \For{i=1}{n}\Comment{sum(i)}
%    \State $sum:=sum+i$;
% \State writeln($sum$);
% \End.
% \end{algorithmic}

\begin{algorithm}
  \caption{Updating transaction state after two operations}
  \label{alg:updateTxState}
  \begin{algorithmic}  	    
    \State $txState \gets$ \Call{Begin Transaction}{}
    \State $id \gets$ $txState.id$ unique id generated for transaction
    \State $item_{1} \gets $ \Call{Update Operation}{id, $k_{1}$, $v_{1}$} \Comment for some key and value
    \State $txState \gets txState$.addItem($item_{1}$)
    \State $item_{2} \gets $ \Call{Insert Operation}{id, $k_{2}$, $v_{2}$}
    \State $txState \gets txState$.addItem($item_{2}$)
    \State $result \gets $ \Call{Commit Transaction}{$txState$}    
  \end{algorithmic}
\end{algorithm}

Appending transaction items is the crucial step of the algorithm \ref{alg:updateTxState}, because client is responsible for keeping track of all performed operations within transaction. Transaction state is assumed to be source of truth about transaction.
Note that, user can be unaware of transaction items, but client i.e. driver has to keep track of transaction items and append them to transaction state.

%Nodes are unaware of modifications done at other nodes. Only client can gather all that information and keep track of it.
 %       Merging two transaction states, simply means concatenating collections of transaction items from two transactions given that transaction id is the same. In terms of functional programming, transaction state is a monoid.

