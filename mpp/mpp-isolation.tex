%!TEX root = ../thesis.tex

\section{Isolation from other transactions}
Transactions must run in isolation from each other and provide read-committed isolation level, thus transactions cannot see each other, nor anyone else can see effects of transactions before it completes with commit or rollback. Expected read-committed isolation is different than the one known from RDBMS, because in RDBMS such isolation holds write-locks for the duration of transaction, therefore if transaction writes then subsequent reads return written values. We want to avoid distributed locks, which decrease performance and cause deadlocks, therefore read-committed isolation in \mpt algorithm does not hold any locks, thus each \selectMessage receives current $v$, as it is in \database.

\subsection{Private transaction storage}
\label{sec:mpp:privateTxStorage}
Each mutation \mutation{k}{v} done in transaction \transaction should be stored on replicas responsible for given key keeping it private for that \transaction. In order to satisfy isolation requirement, \emph{private transaction storage} has to be present on each node.

Each mutation message is executed by \node{i}, but instead of operating on current live data of \database, mutations \mutations are put into private storage presented in Definition \ref{def:privateTransactionStorage}. 

\begin{definition}
\label{def:privateTransactionStorage}
\emph{Private transaction storage} \txStorage is a local data structure on each node \node{i} which stores mutations \mutations of transaction \transaction in isolation from other transactions \transactions and live data set of database \database and provides operations listed below: 
  \begin{itemize}
    \item $\oplus(\text{\txState}, \text{\mutation{k}{v}}) \mapsto \text{\txItemi{k}}$ -- stores mutation \mutation{k}{v} of transaction \transactionj represented by transaction state \txState and returns transaction item \txItemi{k}. 
    \item $\mathit{clear}(\text{\txState})$ -- removes all mutations \mutations associated with transaction \transactionj represented by transaction state \txState on node \node{i}
    \item $\mathit{get}(\text{\txState}) \mapsto \text{\mutations}$ -- returns set of mutations of transaction \transactionj represented by transaction state \txState stored on replica \node{i} 
  \end{itemize}
\end{definition}

Definition \ref{def:privateTransactionStorage} abstracts from memory and time, but an implementation of private transaction storage has to $\mathit{clear}(\text{\txState})$ after certain timeout in order to free memory (assuming in-memory storage) and to be resilient to failures of clients, which can fail and abandon transaction \transactionj at any point in time. From perspective of the algorithm it does not matter whether mutations \mutations are stored in memory or in any other way as long as \mutations are accessible during transaction's execution. 


\subsection{Quorum requirements}
Each mutation \mutation{k}{v} of transaction \transactionj should be written to at least quorum of replicas $N^{\mathbb{RF}} = \tau(k)$ (see Definition \ref{def:topology}) in order to depend on majority during the commit process. If quorum is not available, then \transactionj needs to be rolled back. 

\subsection{Isolated operations}
Algorithm \ref{alg:transactionalInsert} shows how private transaction storage is used in relation to \insertMessage. Note that all other mutation messages are executed in the same way. 

\begin{algorithm}

\algblockdefx[NAME]{Message}{EndMessage}%
   [1][Unknown]{\textbf{Receive} \emph{#1} \textbf{message}}%
   {\textbf{end}}   

\algblockdefx[NAME]{ForEachReplica}{EndForEachReplica}%
   {\textbf{for each} \emph{n} $\gets$ $\mathit{N^{\mathbb{RF}}}$}%
   {\textbf{end}}   

  \caption{Transactional upsert}
  \label{alg:transactionalInsert}
  \begin{algorithmic}
    \Message[\insertMessage]      
      \State $\mathit{N^{\mathbb{RF}}} \gets \tau(k)$      
      \ForEachReplica
        \State send $\mathit{M}(n_{i}, n, \mathit{upsert}(\text{\txState}, k,v))$
      \EndForEachReplica
      \State $\mathit{acks} \gets $ await $\mathit{ack}(\lambda_{k})$ messages
      \If {$\mathit{quorum(acks)}$} 
        \State send $\mathit{M}(n_{i}, c, \mathit{update\_transaction\_state}(\lambda_{k}))$        
      \Else
        \State send $\mathit{M}(n_{i}, c, \mathit{failure}(\mathit{upsert}(\text{\txState},k,v)))$        
      \EndIf
    \EndMessage
  
  \end{algorithmic}
  
  \begin{algorithmic}
    \Message[$\mathit{M}(n_{i}, n, \mathit{upsert(\text{\txState},k,v)})$]
        \State $\lambda_{k} \gets \oplus(\text{\txState}, \text{\mutation{k}{v}})$
        \State send $\mathit{M}(n, n_{i}, \mathit{ack}(\lambda_{k}))$
    \EndMessage
  \end{algorithmic}
\end{algorithm}
 
 
