%!TEX root = ../thesis.tex


\begin{figure}[H]
  \centering
	\subfloat[The commit procedure]{%
      \setlength{\unitlength}{0.8cm}
      \begin{sequencediagram} 
  	\newthread[green!30	]{c}{\client} 
  	\newthread{ni}{\node{i}}
  	\tikzstyle{inststyle}+=[bottom color=blue!30,
		top color=blue!30, rounded corners=3mm]
  	\newinst[2]{nj}{\node{j}}
  	\newinst{nk}{\node{k}}
  	\newinst{nl}{\node{l}}

	\mess{c}{ tx commit }{ni}

	\begin{sdblock}{The commit procedure}{loop of transition messages}
		% \mess{ni}{  }{nj}
		% \mess{ni}{  }{nk}
		% \mess{ni}{  }{nl}

		% \mess{nj}{  }{ni}
		% \mess{nk}{  }{ni}
		% \mess{nl}{  }{ni}
		\begin{callself}{ni}{next transition}{} 
		\postlevel
		\begin{call}{ni}{transition()}{nj}{}
		\end{call}
		\begin{call}{ni}{transition()}{nk}{}
		\end{call}
		\begin{call}{ni}{transition()}{nl}{}
		\end{call}
		\end{callself}
		
	\end{sdblock}
	\mess{ni}{ tx commit response }{c}
	
	\end{sequencediagram}
    } \par
    \subfloat[Message labels]{      
      \begin{tabular}{|p{3cm}|p{9cm}|}
        \toprule
        label & description \\ \midrule
        tx commit & \client wants to commit transaction \transaction and sends message \txCommitMessage \\ 
        transition() & \node{i} sends transition messages to nodes $n\in \text{\nodesTx}$ \\ 
        tx commit response & \node{i} responds with commit results in message \txCommitResonseMessage \\ \bottomrule
      \end{tabular}
    }
  \caption{Start of the commit procedure}
  \label{fig:seqCommitBasic}
\end{figure}