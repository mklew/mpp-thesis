%!TEX root = ../thesis.tex

\section{The requirements}\label{sec:mpp:requirements}

\subsection{Paxos requirements}
Paxos defines requirements and constraints that the protocol must operate within \cite{lamport2001paxosMadeSimple}. Some of them are \begin{enumerate*}
  \item Every proposal has a sequence number, that can be generated by any proposer
  \item Majority has to promise to a proposer, before proposer can make proposal
  \item Acceptors reply with value associated with highest numbered proposal that they have already accepted. Then proposer is bound to ask acceptors to only commit that value (which might be different from its original proposal). -- this forces proposer to commit other completed proposals instead of his own.
  \item Proposer can finish protocol only when majority of acceptors acknowledge commit.
\end{enumerate*}
These are requirements that need to be satisfied in order to be sure that \paxos can work the way it was designed and proven formally. 

\lwt supports many instances of paxos protocol at any given time, as opposed to appending to single replicated log \cite{chandra2007PaxosMadeLive}. \mpp has to run many instances of paxos protocol as well. Thus \mpp needs to have ability to identify \emph{paxos round} at node given proposed \emph{value}. \lwt identifies \emph{paxos round} by a partition key. \mpp is a multi-partition protocol, therefore identification of \emph{paxos round} poses a problem and a requirement.

% TODO dac referencje na to skad te wymagania sie wziely

%1. [a]
%2. Ability to finish paxos round given in progress proposal.
%3. 
%4. Quorum of nodes has to accept proposal before commit
%5. Quorum of nodes has to acknowledge commit before leader can finish paxos round


% TODO tak samo napisac skad te wymagania sie wziely
\subsection{Transaction requirements}
Transactions, especially known from \emph{RDBM} systems (\ref{sec:theory:transactions}), have well defined behaviour in \emph{ACID} terms. Transactions are expected to have following functionality:
\begin{enumerate*}
\item isolation from other transactions. Transaction's state and data should be isolated from other transaction's state and data.
\item serializability -- if two transactions modify same piece of data and commit at same time, only one transaction can be committed, as a corollary second has to be rolled back.
\item begin transaction operation -- transaction is begun, subsequent operations are part of the transaction. 
\item commit transaction operation -- as a result, transaction is either committed or rolled back.
\item rollback transaction operation -- rollback whole transaction and its modification if needed
\item do an operation within transaction - any data manipulation operation should be possible to execute in context of transaction
\item atomicity
\item durability
\item consistency -- in case of \mpp, eventual consistency.
\end{enumerate*}

Proposed \mpp holds all imposed requirements. Following sections address requirements in context of \mpp.

%Transactions should have following functionality:
TODO napisac czym sie roznica od LWT - dodatki do prywatne dane - to juz opisalem w The advantage \\
TODO napisac czym sa memtable - tylko pytanie gdzie to napisac, to moze byc opisane w innej sekcji \\


%1. Run in isolation from other transactions - transactions - their state and data - should be isolated from each other. One transaction cannot see other’s data.
%2. Serializability - if two transactions modify same piece of data and commit at same time, only one transaction can be committed, as a corollary second has to be rolled back.
%3. Begin transaction - that how any transaction is started in relational databases. Transaction is begun and then 
%4. Commit transaction - once transaction’s logic has finished, user commits transaction and sees either successful commit or rollback of transaction. 
%5. Rollback transaction - rollback whole transaction if needed
%6. Do an operation within transaction - any data manipulation operation should be possible to execute in context of transaction
%7. Atomicity
%8. Durability
%9. Eventual consistency


%        Requirements summary
%All of requirements above were addressed in algorithm. There is an solution to each requirement in next sections followed by summary of whole algorithm.


TODO jakiś diagram który pokazuje o co chodzi, po co to jest np. Diagram z nodes i transakcjami.
