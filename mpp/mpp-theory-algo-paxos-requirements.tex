%!TEX root = ../thesis.tex

\section{Consensus by Paxos}
We use \paxos to reach consensus among nodes \nodesTx, which select a single transaction \transaction to commit out of set of concurrent \emph{conflicting} transactions \transactions (Definition \ref{def:conflictingTransactionsSet}). \paxos provides consensus on a single value among many values proposed during the same \paxos round, therefore in order to commit a single transaction \transaction, all conflicting \transactions must participate in the same \paxos round, which is not trivial considering many mutations \mutationsFull, as there is no longer single key $k$, which would identify \paxos round, as it is in \lwt.

\begin{definition}
\label{def:conflictingTransactionsSet}
\emph{Conflicting transactions set} - denoted by \conflictingTxSet is a set of transactions where all \transactions include at least single common mutation $\delta(k)$, which mutates value associated with the same key $k$.
\end{definition}

\subsection{Proposed transaction state}
Our true value is transaction \transactionFull, however mutations \mutationsFull are distributed over nodes
\nodesOfMutations. As a result of that it is impossible to propose \transaction itself, but it is possible to propose transaction
state \txState, which references mutations \mutations, thus proposed \paxos \emph{value} in \mpt is \txState. Any
node \node{i} can try to commit \transaction given \txState, because \node{i} can check which nodes participate in \transaction and where are private data stored.


\subsection{Reaching the same Paxos round}
Conflicting transactions \conflictingTxSet must participate in the same \paxos round identified by \paxos round id \paxosRoundId.
If there are many transactions \transactions being committed at the same time and those \transactions are in conflict
with each other, then only single transaction \transaction should get committed and rest of them should be rolled back. Nodes \nodesTx must agree on \paxos value, which is transaction state \txState, thus agree on which \transaction is committed.

Rest of transactions $(\mathcal{C}\text{\txStates} - \text{\txState})$ can be rolled back, as long as they participate in the same \paxos round \paxosRoundId. In case of \lwt \paxosRoundId is determined by \emph{k}, however \mpt supports more than one key, therefore we need a function which maps \txState $\mapsto $ \paxosRoundId with properties: 
\begin{enumerate*}[label=\alph*)]
\item it maps conflicting transactions to the same \paxosRoundId,
\item it maps non-conflicting transactions to different $\iota'$.
\end{enumerate*}

\subsection{Conflict function}
Function \txState $\mapsto $ \paxosRoundId can be a composition of two functions: \mbox{\txState $\mapsto $ \conflictingTxSet} and \mbox{\conflictingTxSet $\mapsto$ \paxosRoundId}, where the former groups transactions into conflicting sets and the latter assigns \paxos round id \paxosRoundId to each set.
In order to group conflicting transactions we need a function, which compares transaction states \txStateOne and \txStateTwo in pairs and detects whether
\txStateOne and \txStateTwo are in conflict. Definition \ref{def:conflictFunction} presents such function.

\begin{definition}
\label{def:conflictFunction}
\emph{Conflict function} denoted, as $\zeta (\text{\txStateOne, \txStateTwo}) \mapsto ( \mathcal{C}_1, \mathcal{C}_2)$, where $\mathcal{C}_1 = \mathcal{C}(\text{\txStateOne, \txStateTwo}) \wedge \mathcal{C}_2 = \emptyset $ or $\mathcal{C}_1 = \mathcal{C}(\text{\txStateOne}) \wedge \mathcal{C}_2=\mathcal{C}(\text{\txStateTwo})$, the former case is when transactions are in conflict and contain at least single $\delta$ for the same $k$, the latter otherwise.
\end{definition}