%!TEX root = ../thesis.tex

\section{The rationale}
The \mpt algorithm is designed for distributed databases with token ring architecture,
which uses \emph{keys} to lookup replicated data in the cluster. However, there are no strict restrictions, thus the algorithm potentially supports any distributed database.

\mpt supports multi-partition operations.
\mpt supports transactions with read-committed isolation level. 
\mpt supports transactions that are atomic and durable.
\mpt is resilient to network partitions. In terms of \emph{CAP} \cite{Brewer:2012ba} it uses \emph{AP}
\mpt is eventually consistent (\ref{sec:theory:eventualConsistency}). 

Comparison of \mpt to \lwt is shown at fig. \ref{fig:mppVsLwt}.

% TODO poprawić, że w MPP jest read-committed a nie serializable.
\begin{figure}[hbt]
  %\centering
  \setlength{\unitlength}{1.3cm}  
  \subfloat{
    \renewcommand{\tabcolsep}{0.1cm}
    \resizebox{\textwidth}{!}{\begin{tabular}{c|c|c|c|c|c|c}
      \toprule
      Algorithm & multi-partition operations & isolation level & atomicity & durability & consistency & \emph{CAP} \\ \midrule
      \mpt      &   yes                      & read-committed  &   yes     & yes        & eventually consistent & \emph{AP}  \\
      \lwt      &   no, single partition     & serializable  &   yes     & yes        & eventually consistent & \emph{AP}  \\  \bottomrule      
    \end{tabular}}
  }
  \caption{\mpt compared to \lwt}
  \label{fig:mppVsLwt}
\end{figure}

%MPPaxos provides serializable transactions that span multiple tokens on token ring. Provides transactions which are isolated from each other. Transactions that are atomic and durable.
%Consistency is eventual for all nodes, but reads are always consistent as long as quorum of nodes is considered. This is the same as in LWT.

% Algorithm’s rationale (theoretical part) - TODO, wstępny akapit przenieść do motywacji



