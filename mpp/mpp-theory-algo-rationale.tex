%!TEX root = ../thesis.tex

\section{The rationale}
The \mpt algorithm is designed for distributed databases with token ring architecture,
which uses \emph{keys} to lookup replicated data in the cluster. However, there are no strict restrictions, thus the algorithm potentially supports any distributed database, its key features are: \begin{enumerate*}
\item supports multi-partition operations,
\item supports transactions with read-committed isolation level,
\item supports transactions that are atomic and durable,
\item is resilient to network partitions. In terms of \emph{CAP} \cite{Brewer:2012ba} it uses \emph{AP}.
\end{enumerate*}
% Nie piszę o eventuall consistency, bo to Cassandra jest eventual consistent, a sam algorytm nie ma do tego środków oprócz tego, że bazuje na Quorums więc jest na pewno AP.
%\item is eventually consistent \ref{sec:theory:eventualConsistency}, since sacrafices .


%MPPaxos provides serializable transactions that span multiple tokens on token ring. Provides transactions which are isolated from each other. Transactions that are atomic and durable.
%Consistency is eventual for all nodes, but reads are always consistent as long as quorum of nodes is considered. This is the same as in LWT.

% Algorithm’s rationale (theoretical part) - TODO, wstępny akapit przenieść do motywacji



