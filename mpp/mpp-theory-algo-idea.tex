%!TEX root = ../thesis.tex

TODO nie pisać o tym, że rozszerzam Paxosa tylko, że używam go.

\mpp also relies on \paxos, but with pieces added and changed, thus extended
\paxos is part of the \mpp algorithm. As a consequence, \mpp has to adhere to \paxos requirements in addition to requirements which are defined for transaction like behaviour. 

\section{The advantage}
TODO przeniesc jako porównanie do LWT

\emph{LWT} \ref{sec:theory:transactions:lwt} is limited to a single partition, thus \lwt cannot be used for atomic updates of many partitions, thereby this paper presents a new algorithm, which overcomes limitations of \lwt.
Multi Partition Paxos (\mpp) is an algorithm that supports multi-partition operations. Since at its core lays \paxos then it has network partition tolerance, therefore is applicable to distributed environments. \mpp provides read-committed isolation level, whereas \lwt provides serializable isolation.