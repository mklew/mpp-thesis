%!TEX root = ../thesis.tex

\subsection{Transaction index}
\label{sec:mpp:txIndex}
Transaction index denoted, as \txIndex is a service, local to a node, which provides function \mbox{\txState $\mapsto $ \paxosRoundId} required to reach the same \paxos round for conflicting transactions.

Registration of transaction state \txState in \txIndex on each \node{i} $\in$\nodesTx is precondition to the commit procedure of \mpt. 
Since \txIndex is local, each transaction \transaction receives different \paxos round id \paxosRoundId on different nodes, but the value of \paxosRoundId is never used outside of a node, thus \paxosRoundId does not have to be globally the same. \paxosRoundId is used only to identify \paxos round for \conflictingTxSet at each \node{i}. If \transaction participates in the same \paxos round identified by \paxosRoundId as other conflicting transactions, then only single \transaction is committed after \nodesTx agree on the proposed \txState.

% TODO może dodać ten paragraf
%Providing same paxos round id would require another consensus algorithm of its own which would inherently increase cost of the whole algorithm, thus it is suitable to avoid as many round-trip requests as possible.

%\paxosRoundId, used to find \paxos State data at each node locally, is a synthetic id and is different at each replica during \mpt round. 

Transaction index stores sets of conflicting transactions \conflictingTxSet and assigns \paxosRoundId to each set. When a new transaction state \txStateM is registered, \txIndex tries to find a set \conflictingTxSet to which \txStateM can be added by applying conflict function \conflictFunction to \txStateM and each transaction \transaction in each set. If there is a conflict, sets can be merged together, thus \txStateM joins \conflictingTxSet and receives the same \paxosRoundId. Registration algorithm is depicted in Algorithm \ref{alg:indexRegistration}.

During the commit procedure of \mpt transaction \transactionFull has to register in each quorum of each replica subset
$(\NRF_{k_1} \cup \NRF_{k_2} \cup ... ) =
(\tau(k_1) \cup \tau(k_2) \cup ... ) = 
 \text{\nodesTx}\in\mathit{N}$.
Moreover transaction \transaction cannot be registered in transaction index \txIndex when it can obtain more than one \paxos round id \paxosRoundId. These two requirements guarantee that if transaction \txOne and transaction \txTwo are conflicting for some key $k$, then there exists node \node{i}
$\in N^{RF}_{k}$ on which \txOne and \txTwo receive the same \paxosRoundId, thus both transactions are part of the same \paxos round and eventually only one is committed.

\begin{algorithm}
\algblockdefx[NAME]{ForEach}{EndForEach}%
   [2]{\textbf{for each} \emph{#1} $ \gets $ \emph{#2}}%
   {\textbf{end}}   

  \caption{Registration of transaction state \txState in transaction index}
  \label{alg:indexRegistration}
  \begin{algorithmic}  	
    \State $\mathit{idx} \gets$ set of pairs $(\text{\conflictingTxSet, \paxosRoundId})$ present in the index
  	\State $\mathit{cc} \gets \emptyset$  \Comment{set of conflicting sets}  
  	\ForEach{ $(\mathcal{C}(\text{\txStatei{1}, \txStatei{2}, ...}),\text{\paxosRoundId})$ }{$\mathit{idx}$}
  		\ForEach{\txStatei{i}}{$\mathcal{C}(\text{\txStatei{1}, \txStatei{2}, ...})$}
      \State $( \mathcal{C}_1, \mathcal{C}_2) \gets \zeta (\text{\txState, \txStatei{i}})$ \Comment{apply confict function}
  		\If {$ (\text{\txState, \txStatei{i}}) \subset \mathcal{C}_1 $}  
        	\State $\mathit{cc} \gets \mathit{cc} \cup (\mathcal{C},\text{\paxosRoundId})$ 
      \EndIf	
      \EndForEach
  	\EndForEach      	
  	\If {$\mathit{cc} \equiv \emptyset$} \Comment{start a new set}
      \State \paxosRoundId $\gets \mathit{random\_id()}$ 
      \State $\mathcal{C}_{new} \gets \mathcal{C}(\text{\txState})$ 
  		\State $\mathit{idx} \gets \mathit{idx} \cup (\mathcal{C}_{new}, \text{\paxosRoundId})$ 
  	\ElsIf {$\mathit{cc} \equiv ((\mathcal{C},\text{\paxosRoundId}))$}
  		\State $\mathcal{C} \gets \mathcal{C} \cup \text{\txState}$  \Comment{add to the set}
  	\Else
  		\State \txState cannot be registered in index
  	\EndIf
  \end{algorithmic}
   
\end{algorithm}

\subsubsection{Example of registration on single \node{i}}
Assume we have transaction states:
\txStatei{1}$(\text{\txItemi{1}},\text{\txItemi{2}}, \text{\txItemi{3}})$,
\txStatei{2}$(\text{\txItemi{2}},\text{\txItemi{4}})$,
\txStatei{3}$(\text{\txItemi{4}})$.
Registration in index is done according to Algorithm \ref{alg:indexRegistration}. 
Two different orderings yield different outcomes. 

 \begin{description}
 \item[Order $\text{\txStatei{1}} \rightarrow \text{\txStatei{2}} \rightarrow \text{\txStatei{3}}$] \hfill \\
 	\txStatei{1} registers in \txIndex and obtains \paxosRoundIdi{1} \\
 	\txStatei{2} registers in \txIndex and obtains \paxosRoundIdi{1} from conflicting transaction \txStatei{1} \\
 	\txStatei{3} registers in \txIndex and obtains \paxosRoundIdi{1} from conflicting transactions: \txStatei{1}, \txStatei{2} All transactions participate in the same \paxos round and only single transaction is committed. 
 \item[Order $\text{\txStatei{1}} \rightarrow \text{\txStatei{3}} \rightarrow \text{\txStatei{2}}$] \hfill \\
 	\txStatei{1} registers in \txIndex and obtains \paxosRoundIdi{1} \\
 	\txStatei{3} registers in \txIndex, there are no conflicts, thus obtains \paxosRoundIdi{3}\\
 	\txStatei{2} registers in \txIndex and it has $2$ conflicting transactions participating in different paxos rounds with ids: \paxosRoundIdi{1}, \paxosRoundIdi{3}. It is illegal to register \txStatei{2}, because if \txStatei{2} is registered in both rounds it can be committed in round \paxosRoundIdi{1}, but rolledback in round \paxosRoundIdi{3} which is a contradiction.

 \end{description}

\subsubsection{Example of registration in quorums}
Assume we have transaction states
\txStatei{1}$(\text{\txItemi{1}},\text{\txItemi{2}}, \text{\txItemi{3}})$,
\txStatei{2}$(\text{\txItemi{2}},\text{\txItemi{3}})$, \RF{3} and $N=5$. Note that both transactions have transaction item \txItemi{2}, thus mutate the same key. Given that \txItem is equivalent of a key (Definition \ref{def:transactionItemChapter3}) we can use topology \topology to find replicas on which transactions have to register. Assume topology \topologyItem{1}{N'} and \topologyItem{2}{N''} where $N^'=(n_1, n_2, n_3)$ and $N^{''}=(n_3,n_4,n_5)$.

Transaction state \txStatei{1} has to register in quorum of $N^'$ and quorum of $N^{''}$; \txStatei{2} has to register in quorum of $N^{''}$.
$quorum(N^{''})$ is equal to one of $((n_3,n_4),(n_3,n_5),(n_4,n_5))$, therefore $\mathit{quorum(N^{''}}) \cap \mathit{quorum(N^{''}}) \neq \emptyset$, thus exists a $n\in N^{''}$ on which boths transactions are registered.
\txStatei{1} and \txStatei{2} receive the same \paxosRoundId on $n$, thus if one is committed, the other has to be rolled back.

\subsection{Rolling back concurrent transactions}
Transaction index knows which \paxosRoundId to assign to which transaction state \txState and it knows set \conflictingTxSet to which \txState is assigned. This knowledge is used to rollback conflicting transactions when \paxos round \paxosRoundId is finished at ndoe \node{i} and \txStateCommitted is learnt by \node{i}. The node calls $\mathit{clear}(\text{\txState})$ function of private transaction storage for each \mbox{\txState $\in$ $(\text{\conflictingTxSet} - \text{\txStateCommitted})$}.

If a client \client wants to rollback transaction \transaction then that client
sends a message \txRollbackMessage after which \node{i} identifies \nodesTx and broadcasts message \rollbackMessage for each \node{j} $\in$ \nodesTx. When \node{j} receives such message it
uses $\mathit{clear}(\text{\txState})$ function of private transaction storage. \transaction is rolled back after all 
\mbox{$n\in \text{\nodesTx}$} clear private data.