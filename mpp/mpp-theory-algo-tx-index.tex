%!TEX root = ../thesis.tex

\subsection{Transactions index}
Paxos round id for a transaction is obtained through \emph{Transactions Index}. This is an index in which transactions have to successfully register. Registration in index of part of \mpp algorithm. Transaction Index exists on each node and is local to a node, thus independent from other nodes.

Transaction Index has to guarantee that: 
\begin{enumerate*}
\item Transactions that are in conflict must be registered in index and obtain same paxos id
\item Transaction can register in index only if there is single paxos id which can be assigned to this transaction.
\end{enumerate*}

In this work, I put forward the claim that \emph{paxos round id} does not have to be globally assigned to same id, but only locally at each node. Therefore, at each node transaction can participate in paxos round with different id. However, at any node, transaction always participates in the same round as conflicting transactions.

Transaction cannot be registered in index when locally it can obtain more than one paxos round id. Algorithm to register is presented at Fig. \ref{alg:indexRegistration}.

\begin{algorithm}
\algblockdefx[NAME]{ForEach}{EndForEach}%
   [2]{\textbf{for each} \emph{#1} $ \gets $ \emph{#2}}%
   {\textbf{end}}   

  \caption{Registration in transaction index}
  \label{alg:indexRegistration}
  \begin{algorithmic}
  	\State $TS \gets$ transaction state of transaction to register
  	\State $rounds \gets $ empty set 
  	\State $items \gets $ subset of $TS.transactionItems$ owned by this node
  	\ForEach{ ti }{TS.transactionItems}
  		\State $conflicting \gets $ \Call{Find Conflicting}{ti} \Comment 
  		\If {conflicting is not empty } 
        	\State add paxos round id from conflicting participants to $rounds$
      \EndIf	
  	\EndForEach    
  	\State $round \gets$ unassigned round id
  	\If {rounds size = 0}
  		\State $round \gets$ generate new round id 
  		\State \Call{Append Transaction}{TS, items, round} 		
  	\ElsIf {rounds size = 1}
  		\State $round \gets$ get round from $rounds$
  		\State \Call{Append Transaction}{TS, items, round}
  	\Else
  		\State transaction cannot be registered in index
  	\EndIf


  	\Procedure{Append Transaction}{$TS, items, round$} 
	  	\ForEach{ ti }{items}
	  		\State $participants \gets$ get participants from $index$ map by key $ti$ 
	  		\State append $(TS, round)$ to participants
	  	\EndForEach       		
	\EndProcedure  	
  \end{algorithmic}
   
\end{algorithm}

\subsubsection{Example of registration}
Assume $3$ transactions: $tx_{1}, tx_{2}, tx_{3}$ and $4$ different transaction items: $ti_{1}, ti_{2}, ti_{3}, ti_{4}$. Assume conflict function on same transaction items.
Assume items in transactions $tx_{1} \rightarrow (ti_{1}, ti_{2}, ti_{3})$, 
 $tx_{2} \rightarrow (ti_{2}, ti_{4})$, $tx_{3} \rightarrow (ti_{4})$
Assume that registration of transactions happens in arbitrary order. Registration in index is done according to algorithm (\ref{alg:indexRegistration}). Two different orderings yield different outcomes. 
 \begin{description}
 \item[Order $tx_{1} \rightarrow tx_{2} \rightarrow tx_{3}$] \hfill \\
 	$tx_{1}$ registers in index and obtains paxos id $id_{1}$ \\
 	$tx_{2}$ registers in index and obtains paxos id from conflicting transaction $tx_{1}$, $id_{1}$ \\
 	$tx_{3}$ registers in index and obtains paxos id $id_{1}$ from conflicting transactions: $tx_{1}, tx_{2}$ All transactions participate in the same paxos round. Only one transaction is chosen during paxos round. 
 \item[Order $tx_{1} \rightarrow tx_{3} \rightarrow tx_{2}$] \hfill \\
 	$tx_{1}$ registers in index and obtains paxos id $id_{1}$ \\
 	$tx_{3}$ registers in index, there are no conflicts, thus obtains paxos id $id_{3}$\\
 	$tx_{2}$ registers in index and it has $2$ conflicting transactions participating in different paxos rounds with ids: $id_{1}$, $id_{3}$. It is illegal to register $tx_{2}$, because if $tx_{2}$ is registered in both rounds it can be accepted in round $id_{1}$, but not accepted in round $id_{3}$ in which case it is committed and rolled back at the same time which is illegal behaviour. 

 \end{description}

% Let’s assume that:
% * There are three transactions represented by transaction states: Tx1, Tx2, Tx3.
% * Tx1 modifies items Ti1, Ti2, Ti3
% * Tx2 modifies items Ti2, Ti4
% * Tx3 modifies items Ti4
% * conflict resolution is defined as conflict on same transaction items.
% * All transactions try to execute at same time


% Order of transactions registering in transaction index can be arbitrary. Let’s analyze two interesting cases
% 1. Tx1 -> Tx2 -> Tx3
% 2. Tx1 -> Tx3 -> Tx2


% Case 1)
% 1. Tx1 registers in index
%    1. There are no conflicting transactions
%    2. New paxos round is started with paxos id PaxosId_1
% 1. Tx2 registers in index
%    1. Tx1 exists in index and has conflict on item Ti2
%    2. Tx2 joins paxos round of Tx1 
%    3. Tx2 receives round id PaxosId_1
% 1. Tx3 registers in index
%    1. There are 2 conflicting transactions
%    2. But both of them belong to same round. 
%    3. Tx3 joins round with id PaxosId_1


% In that case, all transactions go into same paxos round, therefore after paxos round, replicas will agree on single transaction. That’s exactly what we wanted and what will guarantee serializability.


% Case 2)
% 1. Tx1 registers in index and receives PaxosId_1
% 2. Tx3 registers in index
%    1. There are no other transactions that modify Ti4
%    2. New paxos round is started with paxos id PaxosId_2
% 1. Tx2 registers in index
%    1. There are 2 conflicting transactions: Tx1 and Tx3
%    2. Transactions belong to different paxos rounds
%    3. Tx2 cannot start and has to rollback or try again.


% What if Tx2 was somehow registered in both rounds? Then it might be a case where Tx2 is committed at one round, but rolled back at the other which is illegal because it breaks serializability. We need to guarantee that two concurrent conflicting transactions will not get committed at same time. Allowing Tx2 to participate in any of these paxos rounds could potentially break such guarantee.


\subsubsection{Rolling back concurrent transactions}
\emph{Transaction Index} knows which paxos round id to assign to which transaction, but also it knows other transactions that share same paxos round id. This knowledge is used to rollback concurrent transactions when paxos round is finished at this node and transaction is committed.


% To summarize, transaction index main responsibility is to assign paxos round id for each transaction in a way that doesn’t break any of requirements. Index itself changes a lot since there are many concurrent transactions that begin, try to commit and at the end are either committed or rolled back. 
