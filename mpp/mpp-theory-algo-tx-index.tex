%!TEX root = ../thesis.tex

\subsection{Transaction index}
\label{sec:mpp:txIndex}
Transaction index denoted, as \txIndex is a service, local to a node, which provides \mbox{\txState $\mapsto $ \paxosRoundId} function required to reach the same \paxos round for conflicting transactions.

Registration of \txState in \txIndex on each \node{i} $\in$\nodesTx is precondition to the commit procedure of \mpt. 
Since \txIndex is local, each \transaction receives different \paxosRoundId on different nodes, but the value of \paxosRoundId is never used outside of a node, thus \paxosRoundId does not have to be globally the same. \paxosRoundId is used only to identify \paxos round for \conflictingTxSet at each \node{i}. If \transaction participates in the same \paxos round identified by \paxosRoundId as other conflicting transactions, then only single \transaction is committed after \nodesTx agree on the proposed \txState.

Transaction index stores sets of \conflictingTxSet and assigns \paxosRoundId to each set. When a new \txStateM is registered, \txIndex tries to find a set \conflictingTxSet to which \txStateM can be added by applying \conflictFunction to \txStateM and each \transaction in each set. If there is a conflict, sets can be merged together, thus \txStateM joins \conflictingTxSet and receives the same \paxosRoundId. Registration algorithm is depicted in Algorithm \ref{alg:indexRegistration}.

During the commit procedure of \mpt \transactionFull has to register in each quorum of each replica subset
$(N^{RF}_{k_1} \cup N^{RF}_{k_2} \cup ... ) =
(\tau(k_1) \cup \tau(k_2) \cup ... ) = 
 \text{\nodesTx}\in\mathit{N}$.
Moreover \transaction cannot be registered in \txIndex when it can obtain more than one \paxosRoundId. These two requirements guarantee that if \txOne and \txTwo are conflicting for some $k$, then there exists \node{i}
$\in N^{RF}_{k}$ on which \txOne and \txTwo receive the same \paxosRoundId, thus both transactions are part of the same \paxos round and eventually only one is committed.

\begin{algorithm}
\algblockdefx[NAME]{ForEach}{EndForEach}%
   [2]{\textbf{for each} \emph{#1} $ \gets $ \emph{#2}}%
   {\textbf{end}}   

  \caption{Registration in transaction index}
  \label{alg:indexRegistration}
  \begin{algorithmic}
  	\State $TS \gets$ transaction state of transaction to register
  	\State $rounds \gets $ empty set 
  	\State $items \gets $ subset of $TS.transactionItems$ owned by this node
  	\ForEach{ ti }{TS.transactionItems}
  		\State $conflicting \gets $ \Call{Find Conflicting}{ti} \Comment 
  		\If {conflicting is not empty } 
        	\State add paxos round id from conflicting participants to $rounds$
      \EndIf	
  	\EndForEach    
  	\State $round \gets$ unassigned round id
  	\If {rounds size = 0}
  		\State $round \gets$ generate new round id 
  		\State \Call{Append Transaction}{TS, items, round} 		
  	\ElsIf {rounds size = 1}
  		\State $round \gets$ get round from $rounds$
  		\State \Call{Append Transaction}{TS, items, round}
  	\Else
  		\State transaction cannot be registered in index
  	\EndIf


  	\Procedure{Append Transaction}{$TS, items, round$} 
	  	\ForEach{ ti }{items}
	  		\State $participants \gets$ get participants from $index$ map by key $ti$ 
	  		\State append $(TS, round)$ to participants
	  	\EndForEach       		
	\EndProcedure  	
  \end{algorithmic}
   
\end{algorithm}

\subsubsection{Example of registration on single \node{i}}
Assume we have 
\txStatei{1}$(\text{\txItemi{1}},\text{\txItemi{2}}, \text{\txItemi{3}})$,
\txStatei{2}$(\text{\txItemi{2}},\text{\txItemi{4}})$,
\txStatei{3}$(\text{\txItemi{4}})$.
Registration in index is done according to Algorithm \ref{alg:indexRegistration}. 
Two different orderings yield different outcomes. 

 \begin{description}
 \item[Order $\text{\txStatei{1}} \rightarrow \text{\txStatei{2}} \rightarrow \text{\txStatei{3}}$] \hfill \\
 	\txStatei{1} registers in \txIndex and obtains \paxosRoundIdi{1} \\
 	\txStatei{2} registers in \txIndex and obtains \paxosRoundIdi{1} from conflicting transaction \txStatei{1} \\
 	\txStatei{3} registers in \txIndex and obtains \paxosRoundIdi{1} from conflicting transactions: \txStatei{1}, \txStatei{2} All transactions participate in the same \paxos round and only single transaction is committed. 
 \item[Order $\text{\txStatei{1}} \rightarrow \text{\txStatei{3}} \rightarrow \text{\txStatei{2}}$] \hfill \\
 	\txStatei{1} registers in \txIndex and obtains \paxosRoundIdi{1} \\
 	\txStatei{3} registers in \txIndex, there are no conflicts, thus obtains \paxosRoundIdi{3}\\
 	\txStatei{2} registers in \txIndex and it has $2$ conflicting transactions participating in different paxos rounds with ids: \paxosRoundIdi{1}, \paxosRoundIdi{3}. It is illegal to register \txStatei{2}, because if \txStatei{2} is registered in both rounds it can be committed in round \paxosRoundIdi{1}, but rolledback in round \paxosRoundIdi{3} which is a contradiction.

 \end{description}

\subsubsection{Example of registration in quorums}
Assume we have 
\txStatei{1}$(\text{\txItemi{1}},\text{\txItemi{2}}, \text{\txItemi{3}})$,
\txStatei{2}$(\text{\txItemi{2}},\text{\txItemi{3}})$, \RF{3} and $N=5$. Note that both transactions have \txItemi{2}, thus mutate same key. Given that \txItem is equivalent of a key (Definition \ref{def:transactionItemChapter3}) we can use \topology to find replicas on which transactions have to register. Assume topology \topologyItem{1}{N'} and \topologyItem{2}{N''} where $N^'=(n_1, n_2, n_3)$ and $N^{''}=(n_3,n_4,n_5)$.

\txStatei{1} has to register in quorum of $N^'$ and quorum of $N^{''}$; \txStatei{2} has to register in quorum of $N^{''}$.
$quorum(N^{''})$ is equal to one of $((n_3,n_4),(n_3,n_5),(n_4,n_5))$, therefore $\mathit{quorum(N^{''}}) \cap \mathit{quorum(N^{''}}) \neq \emptyset$, thus exists a $n\in N^{''}$ on which boths transactions are registered.
\txStatei{1} and \txStatei{2} receive the same \paxosRoundId on $n$, thus if one is committed, the other has to be rolled back.

\subsection{Rolling back concurrent transactions}
Transaction index knows which \paxosRoundId to assign to which \txState and it knows set \conflictingTxSet to which \txState is assigned. This knowledge is used to rollback conflicting transactions when \paxos round \paxosRoundId is finished at \node{i} and \txStateCommitted is learnt by \node{i}. The node calls $\mathit{clear}(\text{\txState})$ function of private transaction storage for each \mbox{\txState $\in$ $(\text{\conflictingTxSet} - \text{\txStateCommitted})$}.

If \client wants to rollback \transaction then \client
sends a message \txRollbackMessage after which \node{i} identifies \nodesTx and broadcasts \rollbackMessage for each \node{j} $\in$ \nodesTx. When \node{j} receives such message it
uses $\mathit{clear}(\text{\txState})$ function of private transaction storage. \transaction is rolled back after all 
\mbox{$n\in \text{\nodesTx}$} clear private data.