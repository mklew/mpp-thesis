%!TEX root = ../thesis.tex


\section{Outline}
Database $\Omega$ consists of key-value pairs $\{(k_{i},v_{j}), (k_{k},v_{m}),...\}, k_i\in\mathit{K}, v_i\in\mathit{V}$ stored by cluster of nodes $\{n_1, n_2, ...\}, n_i\in\mathit{N}$.
Changes to data are represented by \emph{mutations} $\delta(k,v)$ (Definition \ref{def:mutation}). 
Each mutation is stored in nodes identified by topology $\tau$ (Definition \ref{def:topology}).
Transaction is a set of mutations $\Delta(\delta_{1}, \delta_{2}, ...)$.

\begin{definition}
  \label{def:mutation}
  \emph{Mutation} denoted, as \mutation{k}{v} is an operation, such as upsert (insert or update)\footnote{distinction between insert and update would require its own transaction} or removal of key-value pair \kv. 
\end{definition}

\begin{definition}
\label{def:topology}
\emph{Topology} is a partitioning function $\tau:\mathit{K} \mapsto \mathit{N}^{\mathbb{RF}}$ which for a given $k$ returns a set of replica nodes of size $\mathbb{RF}$. 
\end{definition}


Client $c_{i}\in\mathit{C}$ is an actor, which performs transactions $\Delta_{1}, \Delta_{2}, ...$ and communicates to a cluster \nodes using messages $\mathit{M}$ (Definition \ref{def:message}). \client begins a transaction \transaction by sending a message \beginTransactionMessage to any node $n\in\mathit{N}$ followed by receiving a message 
\initialTxStateMessage, which includes 
initial \emph{transaction state} (Definition \ref{def:transactionStateChapter3}), which will reflect all subsequent changes done in \transaction.

During the transaction \transaction, client \client sends messages of two types:
 mutations and selects \selectMessage. 
 All mutations within a transaction \transaction are isolated from other transactions - they are private to the enclosing transaction until it is committed.
For the duration of the transaction,
mutations are stored in a \emph{private transaction storage}, which is a local data structure at each \node{i}. Select operations query against current state of the database, thus transactions provide read-committed isolation level.

After each mutation client \client receives a message \updateTxStateMessage, which includes a transaction item \txItem referencing that mutation. A client is responsible for tracking changes performed in a transaction, thus it has to append all the received transaction items to the initial transaction state \txStatei{0}.


\begin{definition}
\label{def:transactionItemChapter3}
\emph{Transaction item}  $\lambda$ is a reference to a single mutation \mutation{k}{v} of transaction \transaction, which indicates that replica nodes of $k$ store \mutation{k}{v} in \emph{private transaction storage}, thus $\lambda$ tracks a single change to $k$ done in \transaction. Transaction item is equivalent\footnote{transaction item is redefined in Chapter 4} of a key $k$.
\end{definition}

\begin{definition}
\label{def:transactionStateChapter3}
\emph{Transaction state} $\Lambda$ is a set of transaction items (Definition \ref{def:transactionItemChapter3}) $\{\lambda_{i}, \lambda_{j}, ...\}$, which reflects changes done in transaction \transaction. Transaction state allows to identify nodes $N' \subset N$, which are affected by \transaction and store its \mutations.
\end{definition}



When client \client finishes its planned operations transaction \transaction can be either committed or rolled back. To perform rollback client sends \txRollbackMessage, and then $n$ sends out rollback messages to other nodes \rollbackMessage, where nodes $\{n_j,...,n_k\}$ are identified by \txItems, which 
makes them remove private storage for that transaction
and mark \transaction as rolled back in 
an another local data structure called \emph{transaction log}.

In order to commit transaction \transaction a client \client sends message \txCommitMessage and subsequently \node{i} becomes the proposer of the transaction and is responsible for orchestrating distributed consensus round among nodes affected by transaction \transaction.
 %referenced by transaction items. 
When consensus is reached each node moves private data of transaction \transaction from private transaction storage to the main storage and marks \transaction, as committed in the transaction log. The client \client receives message \txCommitResonseMessage about successfully committed transaction.

There can be many concurrent transactions and some of them mutate values associated with the same keys, thus concurrency control guaranteeing that interfering transactions are not committed at the same time is required. Commit procedure solves this problem by grouping such transactions and performing distributed consensus round, which selects a single transaction that is committed and rollbacks the others.
Commit procedure uses two local data structures \emph{transaction index}, which is responsible for identifing interfering transactions and grouping them into the same consenus rounds and \emph{transaction log}, which records committed or rolled back transactions.

\begin{definition}
	\label{def:message}
	\emph{Message} is defined by $\mathit{M}(n_{i}, n_{j}, \psi)$ where 
	\\ $n_{i}\in\mathit{N}\cup\mathit{C}, n_{j}\in\mathit{N}\cup\mathit{C}$, $\{n_{i}, n_{j}\}\cap\mathit{N}\neq\emptyset$, $n_{i}$ is sender of a message, $n_{j}$ is recipient of a message, $\psi$ is one of payloads listed below:
	\begin{itemize}	
	\item $\mathit{begin\_transaction}()$ -- begins a transaction 
	\item $\mathit{initial\_transaction\_state}(\Lambda_0)$ -- message in response to $\mathit{begin\_transaction}$ message where $\Lambda_0$ is initial \emph{transaction state} (Definition \ref{def:transactionStateChapter3}) 
	\item $\mathit{select(k)}$ -- is a selection of the value associated with key $k$
	\item $\mathit{data(k,v)}$ -- is a message in response to $\mathit{select(k)}$ with $v$ associated with $k$, where $k\in\mathit{K}, v\in\mathit{V}$ 
	\item $\mathit{upsert}(\Lambda,k,v)$ -- is an upsert (insert of update) of key-value pair 
	\item $\mathit{delete}(\Lambda,k)$ -- is a delete of any $v$ associated with $k$
	\item $\mathit{ack}(\lambda)$ -- is an acknowledge of upsert or delete
	\item $\mathit{failure}(\psi)$ -- denotes failure of payload $\psi$
	\item $\mathit{transaction\_commit}(\Lambda)$ -- is a commit of a transaction, where $\Lambda$ is final transaction state 
	\item $\mathit{transaction\_commit\_response}(committed)$ -- is a response to commit $\mathit{transaction\_commit}(\Lambda)$ message, where $committed$ is a boolean value which indicates whether transaction was committed or rolledback
	\item $\mathit{transaction\_rollback}(\Lambda)$ -- is a rollback of a transaction, where $\Lambda$ is final transaction state 
	\item $\mathit{setup}(\Lambda)$ -- inititates \emph{setup phase} of \mpt algorithm 
	\item $\mathit{prepare}(\Lambda, \beta)$ -- initates \emph{prepare phase} of \mpt algorithm; $\Lambda$ is final transaction state, $\beta$ is \paxos ballot 
	\item $\mathit{promise(promised, log, inProgress_{\Lambda})}$ -- is a response to $\mathit{prepare}$ message, where $\mathit{promised}$ is a boolean, $\mathit{log}$ is state of \emph{transaction log} (see Section \ref{sec:mpp:transactionLog}), $\mathit{inProgress_{\Lambda}}$ is accepted but not yet committed transaction state 
	\item $\mathit{propose}(\Lambda, \beta)$ -- initates \emph{propose phase} of \mpt algorithm
	\item $\mathit{propose\_response}(accepted, log)$ -- is a response to $\mathit{propose}$, where $\mathit{accepted}$ is boolean, $\mathit{log}$ is state of transaction log 
	\item $\mathit{commit}(\Lambda)$ -- initates \emph{commit phase} of \mpt algorithm, where $\Lambda$ is transaction state 
	\item $\mathit{commit\_ack()}$ -- is an acknowledge of $\mathit{commit}$ message 
	\item $\mathit{rollback}(\Lambda)$ -- initiates \emph{rollback phase} of \mpt algorithm
	\item $\mathit{update\_transaction\_state}(\lambda)$ -- is a message in response to $\mathit{insert, update, delete}$ messages where $\lambda$ is transaction item (Definition \ref{def:transactionItemChapter3}) 
	\end{itemize}
\end{definition}



