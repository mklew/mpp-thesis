\documentclass[a4paper,onecolumn,oneside,12pt]{report}

% Uwagi
% TODO wszędzie przed AS przecinek

%\usepackage[utf8]{inputenc}

%\usepackage[OT4]{fontenc}
%\usepackage{lmodern}

\usepackage[utf8]{inputenc}
\usepackage[T1]{fontenc}
\usepackage{glossaries}
\usepackage{todonotes}
\usepackage{amsthm}
\usepackage{amsmath}
\usepackage{amssymb}
\usepackage{amsfonts}
\usepackage{paralist}
\usepackage{mathtools}
\usepackage{mathrsfs}
\usepackage{breqn}
\usepackage{subfig}
\usepackage{color}
\usepackage{epic}
\usepackage{eepic}
\usepackage{pstricks}
\usepackage{pst-node}
\usepackage{times}
\usepackage{pstcol}
\usepackage{pst-plot}
\usepackage{multirow}
\usepackage{url}
\usepackage{caption}
\usepackage{float}
\usepackage{tabularx}
\usepackage{algpseudocode}
\usepackage[chapter]{algorithm}
\usepackage{graphicx}
\usepackage{pdfpages}
\usepackage{lscape}
\usepackage{array}
\usepackage{booktabs}

\author{Marek Lewandowski}
\title{Scalable distributed transactions}

\newtheorem{definition}{Definition}[chapter]
\newtheorem{theorem}{Theorem}[chapter]
\newtheorem{lemma}{Lemma}[chapter]

\newcommand{\code}[1]{\texttt{#1}}

\newcounter{ExampleCount}
\setcounter{ExampleCount}{0}
\newenvironment{example}
{ \stepcounter{ExampleCount} {\bf\small Example} \arabic{ExampleCount} \\ }
{  }


\definecolor{vertexColor}{rgb}{0.88, 0.88, 0.88}
\definecolor{vertexBorderColor}{rgb}{0, 0, 0}
\definecolor{vertexTextColor}{rgb}{0, 0, 0}
\definecolor{connectorColor}{rgb}{0.39, 0.51, 0.73}
\definecolor{highlightedConnectorColor}{rgb}{1.0, 0.0, 0.0}

\definecolor{areaColor1}{rgb}{0.5, 0.9, 0.3}
\definecolor{areaColor2}{rgb}{1.0, 0.7, 0.7}
\definecolor{areaColor3}{rgb}{0.6, 0.6, 1.0}
\definecolor{areaColor4}{rgb}{1.0, 0.9, 0.7}
\definecolor{areaColor5}{rgb}{0.7, 0.7, 0.7}
\definecolor{areaColor6}{rgb}{0.9, 0.9, 0.9}
\definecolor{areaColorMix13}{rgb}{0.55, 0.75, 0.65}
\definecolor{areaColorMix34}{rgb}{0.8, 0.75, 0.85}
\definecolor{areaColorMix14}{rgb}{0.75, 0.9, 0.5}
\definecolor{areaColorMix134}{rgb}{0.7, 0.8, 0.67}


\hyphenpenalty=10000      % nie dziel wyrazów zbyt często
\clubpenalty=10000        % kara za sierotki
\widowpenalty=10000       % nie pozostawiaj wdów
\brokenpenalty=10000      % nie dziel wyrazów między stronami
\exhyphenpenalty=999999   % nie dziel słów z myślnikiem
\righthyphenmin=3         % dziel minimum 3 litery

\tolerance=4500
\pretolerance=250
\hfuzz=1.5pt
\hbadness=1450
\sloppy                   % umacnia pozycję prawego marginesu

\linespread{1.3}\selectfont

\begin{document}

%\begin{titlepage}
  \begin{center}
    \fontsize{28pt}{34pt}\selectfont
      WARSAW UNIVERSITY \\
      OF TECHNOLOGY \\

    \vspace*{.5\baselineskip}
    \fontseries{b}\fontsize{24pt}{18pt}\selectfont
      Faculty of Electronics\\ and Information Technology

    \vspace*{4\baselineskip}
    \fontseries{m}\fontsize{32pt}{20pt}\selectfont
      Ph.D. THESIS \\
  
    \vspace*{\baselineskip}
    \fontsize{20pt}{15pt}\selectfont
      Jacek Lewandowski, M.Sc.\\
  
    \vspace*{\baselineskip}
    \fontseries{b}\fontsize{15pt}{18pt}\selectfont
      A hybrid method of indexing multiple-inheritance hierarchies  \\
  
  \end{center}

  \vspace*{6\baselineskip}
  \begin{flushright}
    \fontseries{m}\fontsize{13pt}{10pt}\selectfont
      Supervisor\\
      Professor Henryk Rybiński, Ph.D., D.Sc.\\
  \end{flushright}

  \vspace*{4\baselineskip}
  \begin{center}
    Warsaw, 2012
  \end{center}
\end{titlepage}
\setcounter{page}{2}

\includepdf[pages=-]{pierwsze_strony}

\newpage
\mbox{}

%%!TEX root = thesis.tex
\newpage
\paragraph{Abstract} 
Transactions in non-relational databases, are a rare feature, but a useful and desirable one if present, such as Light Weight Transactions (\lwt) in Cassandra.
The problem with \lwt is lack of a multi partition operations, which limits the practical applications, and puts the burden of maintaining the data consistency in other ways on the users.
The proposed multi-partition transactions algorithm supports transactions spanning many rows with read-committed isolation, and has a scalable design without a single point of failure.
We achieved this by using \paxos algorithm for the distributed consensus, private memtables for isolation, and other techniques, which reduce memory footprint, and increase scalability.
We implemented proof of concept in Cassandra, tested it in a cluster, and confirmed correctness of the algorithm.

\paragraph{Keywords:} paxos, distributed transactions, cassandra, multi-partition transactions

\begin{center}
%\vspace*{\baselineskip}
    \fontseries{b}\fontsize{12pt}{14pt}\selectfont
      Transakcje dla wielu partycji w Cassandrze  \\
\end{center}
\paragraph{Streszczenie} 
Transakcje w nie relacyjnych bazach danych występują rzadko, ale są przydatne i porządne przykładem czego są \emph{Lekkie transakcje} (\lwt) w Cassandrze.
Problem w \lwt to brak wsparcia operacji dla wielu partycji, który ograniczna praktyczne zastosowania i przenosi odpowiedzialność za dbanie o spójność danych na końcowych użytkowników.
Proponowany algorytm wspiera operacje dla wielu partycji zapewniając izolację na poziomie \emph{read-committed} jak również skalowalność systemu oraz brak pojedynczego punktu awarii.
Osiągneliśmy te właściwości używając algorytmu Paxos do rozproszonego konsensusu, prywatnych tabel dla izolacji i innych metod, które obniżają zużycie pamięci i zwiększają skalowalność.
Zaimplementowaliśmy algorytm w Cassandrze, przetestowaliśmy go w klastrze i potwierdziliśmy jego poprawność.

\paragraph{Słowa kluczowe:} paxos, rozproszone transakcje, cassandra, transakcje dla wielu partycji

\pagenumbering{gobble}


\tableofcontents

\section{Data structures}\label{sec:basics:dataStructures}

Data structures we are discussing in this work may be viewed in the contexts of their applications and theory --
especially the lattice theory and the graph theory. We may consider different issues connected with management and
utilization of these structures, in all of the areas mentioned above. Therefore, we will sketch these contexts in the
next paragraphs. We will also set some terms that we will use in the rest of the work.

\subsection{The area of applications}
In the area of applications, the studied data structures are commonly used as hierarchies of different types, especially
as inheritance hierarchies. A characteristic feature of them is an existence of a special element, called a root. The
rest of elements are connected to the root directly or through other elements. Depending on what sort of hierarchy we
are exploring, various sets of terms are possible to describe specific elements. For example, in case of employees
hierarchy, each element represents an employee. An employee's chief is its predecessor in the hierarchy whilst
employee's subordinates are its successors. In the area of object-oriented programming languages and ontologies,
predecessors and successors are called super-classes and sub-classes respectively \footnote{In the area of ontologies
there are yet other terms that are frequently used. They are: concept, super-concept and sub-concept.}, whereas a
connection between a pair of elements is called inheritance or ``is-a'' relation. Another terminology is  depicted on
Fig. \ref{fig:hierarchyElements}.

%%!TEX root = ../thesis.tex

\chapter{Introduction}\label{chapter:introduction}

\section{Preface}\label{sec:introduction:preface}
\emph{Big Data} is the term used for describing sets of frequently changing and increasing volumes of data, which do not fit in a single machine. There are many sources of Big Data, such as system logs, user website clicks, financial transactions, weather measurements, data from Internet of Things, and many others. 

NoSql databases were created to support Big Data and provide means to analyze it. Databases differ in a ways they represent the data, but the key principle stays the same: store the data for future analysis.

NoSql databases span over hundreds, even thousands servers located in different data centers placed around the world connected with internet cable. Number of servers and unreliable network connection presents a problem -- hardware and network failures.
NoSql databases are designed to work under presence of such failures, but it comes with a cost.

\section{Motivation}\label{sec:introduction:motivation}


Properties of relational databases, such as ACID compliant transactions were sacrificed for availability, scalability and fault tolerance. Multi-partition transactions facilitate the work of developers, since they provide clear reasoning about sets of modifications to the data. These transactions do not exist in Cassandra, but also other NoSql databases lack proper transactions support.
Our motivation is to deliver clear reasoning about the data changes, and in order to achieve this the new algorithm was created.

% TODO czemu transakcje są potrzebne
%Transactions are important for performing operations on different entities and ensuring consistency among them, as well as for other reasons. They simplify making changes to the model, and provide guarantees on top of which applications are built. Without transactions, all guarantees disappear, and suddenly developers have to take care of things which were taken for granted.

\subsection{Current state of transactions in Cassandra}	
Cassandra is a distributed database with availability guarantees, which makes it impossible to implement full ACID transactions, however it supports Lightweight Transactions (\lwt).

Listing \ref{lst:intro:lwtInsert} presents \lwt, which inserts some user called \emph{John} only if it is not already present in the database. \lwt  guarantees that if another client inserts another user to the same row at the same time, then only one operation will succeed. In other words, \lwt ensures that steps of a single operation\footnote{(1) checking if row exists, (2) applying an update} are not interleaved with steps of another operation against the same row, when both operations are running around the same time. Therefore \emph{John} is inserted only once, by one or the other transaction.

The difference between \lwt and normal modification operation is that \lwt is a compare-and-set (CAS) operation, thus it has additional \emph{compare} step, before it issues any \emph{set} operation.
The \emph{compare} step of \lwt guarantees that any operations performed for the same key obey to the condition specified in the \code{IF} clause, since conditions are evaluated on the consistent state of the data, as if statements were executed one after the other, thus provides serializability of modifications.

Listing \ref{lst:intro:lwtUpdate} presents an update transaction, which executes only if the \code{balance} column has the expected value of $2000$. If another transaction changes the balance to value other than $2000$ then the former transaction will abort execution after compare step. The conditions in the \code{IF} clause can only refer to the columns in the row that is being modified.

\lwt transaction spans only a single partition, which is a clustering of rows in  table, thus modification to two or more partitions are done in independent \lwt. In terms of ACID, \lwt provides serial isolation level, is atomic and durable. 

\begin{example}
\label{lst:intro:lwtInsert}
\begin{lstlisting}[style=outcode,caption={LWT Insert with \code{IF NOT EXISTS} clause}]
INSERT INTO users (user_id, name, email)  
VALUES (1, 'John', 'john@yahoo.com') 
IF NOT EXISTS
\end{lstlisting}
\end{example}

\begin{example}
\label{lst:intro:lwtUpdate}
\begin{lstlisting}[style=outcode,caption={LWT Update with column condition}]
UPDATE balances 
SET balance = 2500 
WHERE user_id = 1
IF balance = 2000;
\end{lstlisting}
\end{example}


\subsection{Workarounds for no ACID}
Atomic multi-partition updates are missing in Cassadra, nevertheless workarounds are available for instance, a dedicated \emph{step} column, which represents current progress of a \emph{transaction} and a sequence of \lwt operations. Although there is no transaction enforced by the database, there is a transactional code in the logic of the client, which updates \emph{step} column on each operation, and checks the condition against that column in the subsequent operation if its value matches expected step. That way the next operation succeeds only if previous was committed. 
Such model requires client's logic to be written in a specific manner that handles \emph{step} column. 

Multi partition transactions would remove burden of ensuring consistency based on flags and would allow users to perform their operations in more natural way without additional restrictions, and changes in the logic. Moreover, if users relied on relational databases for specific multi-row transactions and used NoSql database for rest of their operations, then all operations would be supported in a single NoSql database instead of mix of the two.

\section{Goals}
The main goal of the thesis is to design an algorithm that provides multi-partition transactions. Moreover it has to be scalable by design and it cannot have a single point of failure. Additionally, the algorithm should be efficient in terms of memory consumption and network usage. The algorithm should be verified and tested for correctness by implementing and testing it in Cassandra. Results, such as scalability and performance should be analyzed.

 
\section{Structure}\label{sec:introduction:structure}
The organization of the work is as follows: 
in the next chapter we introduce concepts of distributed databases followed by ways of implementing relational and distributed transactions including different algorithms, concluding the chapter with descriptions of the databases with main focus on Cassandra. 
The proposed algorithm is discussed in Chapter 3 and its experimental implementation in Chapter 4.
Chapter 5 presents tests and theoretical analysis of the algorithm.
We recapitulate the work in Chapter 6.

%!TEX root = ../thesis.tex

\section{Terms and definitions}\label{sec:introduction:terms}

N -- replication factor \\
node -- something else \\
cluster -- something else else \\
3PC -- 3 phase commit 

%\input{basics/basics.tex}

%\input{related/related.tex}

%\input{hybrid-method/hm.tex}

%%!TEX root = ../thesis.tex

\chapter{Tests and performance analysis}\label{chapter:testing}

Testing chapter is focused on explaining theory behind testing of \mpp algorithm and its components that gives high certainty about correctness of algorithm and its implementation. I’ve run many tests on different levels that had to verify if implementation is correct, but what is more important if new multi partition paxos algorithm delivers its promise of multi--partition serializable transactions. 
Paxos algorithm has formal proof, whereas \mpp algorithm developed during this research has been proven to work in practice for all written tests. Writing a formal proof is a complex task which would require further research and is out of scope of this work.

\section{Unit testing}
Many aspects of algorithm could be and were unit tested. Components such as \emph{Transaction Index}, \emph{Transaction Log}, \emph{Private Transaction Storage} were  tested for their expected behaviour. Tests also covered parts of \mpp which were not cluster dependent such as \emph{phase} transitions. Unit tests are not enough to check for correctness especially when testing distributed algorithms which are  exposed to network latencies, concurrency issues like deadlocks and other problems in the distributed environement. 

\section{Cluster tests}
Tests on real cluster of nodes were major part of all tests performed to validate \mpp correctness. I have conducted multiple experiments which covered all aspects of \mpp algorithm. 

\subsection{Testing environement}
Cluster tests were performed independently on two clusters with $3$ and $5$ nodes per cluster. Tests run on $5$ nodes yielded important results, because with $5$ nodes and $N=3$ there are at most $10$ replica groups. 

Clusters were run locally using \emph{Cassandra Cluster Manager} which is a tool to quickly set up local cluster.

\subsection{Testing method}

\subsubsection{Counters}
Tests are performed using testing schema with counter tables. Counter table is shown on Fig. (\ref{fig:counterTable}). It has $5$ columns with \emph{int} counter values: $c1,c2,c3,c4,c5$
Testing schema consists of $2$ keyspaces with number of counter tables in each. Counter tables in each keyspace differ in name, but structure is the same.

\begin{figure}[h]
\centering
\begin{tabular}{c||c|c|c|c|c}
        \toprule
        id 		 & $c_{1}$ & $c_{2}$ & $c_{3}$ & $c_{4}$ & $c_{5}$ \\ \midrule
        $counter_{1}$ & 1  & 1  & 1  &  1 & 1  \\
        $counter_{2}$ & 1  & 2  & 3  &  4 & 5  \\ \bottomrule
      \end{tabular}
      \caption{A counter table with two sample rows}
  \label{fig:counterTable}
\end{figure}

\subsubsection{Iterations}
Each test case is done in many iterations i.e. $300$. During each iteration, there are many concurrent transactions which execute at the same time. Each transaction performs operations on counters. Most basic operation is to increment counter column using transactional update operation.

\subsubsection{Counter executors}
Counter executors run in parallel through all iterations. 
Counter executors operate on set of counters. They perform same operation on each counter in each iteration, where each iteration is a new transaction that is committed at the end of iteration. 
Basic executor increments specific counter column $c$ of each counter in the set.

\subsection{Testing of repair of in progress proposal}
Tests required modification of an implementation. During this test, counters from table named \code{stop_after_proposed} are incremented. \mpp has been modified to fail after algorithm reached has successfully proposed, thus before commit. It leaves transaction as accepted in progress proposal that should be finished by next proposer. 
Normal select is used to check that changes were not yet committed by the transaction. 
Then a transactional select is used which has to run \mpp algorithm until \emph{prepare phase} is reached. In progress proposal is detected during prepare and has to be completed. Transactional select commits transaction and returns incremented column. Results of select statement are checked for correctness. 

\subsection{Testing of independent transactions}
Tests employs multiple counter executors which receive disjoint sets of counters. Each executor increments counters by one and commits transactions. Each executor keeps track of number of committed transactions.

After iterations, number of committed transactions must match counter values. In case of non conflicting counters, all transactions need to commit, thus counters must have counts equal to number of iterations.

\subsection{Testing of conflicting transactions}
There are $5$ counter executors which receive same set of counters. Each executor increments different count column. 

After iterations, number of committed transactions must match counter values. In case of conflicting transactions counts are less than number of iterations. Counters may have different counts in columns, but all counter rows must have same values in $c_{x}$ columns, because they were incremented by single executor.

\subsection{Testing of partially conflicting transactions}
Test runs the same way as test for conflicting transactions, but executors receive additional disjoint sets of counters, so each executor has intersection on counters plus additional ones private for executor. 

Assertions are the same as in the previous test.

\subsubsection{Different conflict functions}
Testing schema has counter tables with different conflict functions set. \mpp has to work for any conflict functions. Testing scenario remains the same. Expected results are also the same.

\subsection{Testing correctness of commit and rollback}
Testing correctness of rollback and commit is the most crucial test, because tests with incrementations are dependent on truth about rollback and commit responses.

For this test, different counter executors were used. There are $2$ different counter executors. Both of them receive set of counters with even number of counters. Size of the set is double number of iterations. All counters start at $0$.

First counter executor, \emph{Next Two Counter Executor}, in each iteration increments next two counters. In iteration $0$, increments $counter_{0}$, $counter_{1}$, in iteration $1$ increments $counter_{2}, counter_{3}$ and so continues until end of iterations. Second counter, \emph{Even Counter Executor}, in each iteration increments all counters which have even index in the set.

Rationale behind test is that when \emph{Next Two Counter Executor} fails to commit and receives rollback, then for this particular iteration counter with odd index should not be incremented by executor, thus must have initial $0$ count. Otherwise transaction should be committed instead of rolled back.

\subsection{Testing during partial failure}
All previously described tests were also run with cluster during partial failure. Node was shutdown and tests were run. Since \mpp tolerates partial failures, as long as quorum is alive and responds to requests, tests yield same results.

\section{Theoretical performance analysis}

% TODO opis teoretyczny od czego zależy wydajność
% liczba partycji
% liczba nodów
% liczba replik
% liczba współbieżnych transakcji 

% głównie tekst + tabelki i diagramy

% Opisać różne przypadki, jak MPP zachowuje się przy wzroście nodów, replik, ilości konfliktujących transkacji

% TODO każdy czynnik rozważyć osobno (pół strony ) i uzasadnić dlaczego się tak zachowuje.

% TODO, co tu opisac?
%\section{Performance testing}




%%!TEX root = ../thesis.tex

\chapter{Conclusion}\label{chapter:summary}

\section{Summary}
Objective of this work was achieved. TODO

\section{Further research}
Conditions
Formal proof
Reducing number of requests
Increasing concurrency by more effective conflict resolution functions




\listoffigures
\listoftables
\listofalgorithms
 
\bibliographystyle{acm}
\bibliography{thesis}

\end{document}