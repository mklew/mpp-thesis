% obrazek po lewej i tabelka po prawej

\begin{figure}[h]
  \centering
    \subfloat[A sample graph with the two slices - $s_0$ marked with green border and $s_1$ marked with red border.]{%
      \setlength{\unitlength}{0.8cm}
      \begin{pspicture}[showgrid=false](8, 8)
  
        \psset{fillcolor=vertexColor, fillstyle=solid}
        \cnodeput(2, 7){a}{\color{vertexTextColor} A}
        \cnodeput(4, 7){b}{\color{vertexTextColor} B}
        \cnodeput(6, 7){c}{\color{vertexTextColor} C}
        \cnodeput(2, 5){d}{\color{vertexTextColor} D}
        \cnodeput(4, 5){e}{\color{vertexTextColor} E}
        \cnodeput(6, 5){f}{\color{vertexTextColor} F}
        \cnodeput(1, 3){g}{\color{vertexTextColor} G}
        \cnodeput(3, 3){h}{\color{vertexTextColor} H}
        \cnodeput(5, 3){i}{\color{vertexTextColor} I}
        \cnodeput(7, 3){j}{\color{vertexTextColor} J}
        \cnodeput(4, 1){k}{\color{vertexTextColor} K}
        
        \psset{linecolor=connectorColor, arrowscale=2.5}
        \ncline{->}{a}{d}
        \ncline{->}{a}{e}
        \ncline{->}{b}{e}
        \ncline{->}{c}{d}
        \ncline{->}{c}{e}
        \ncline{->}{c}{f}
        \ncline{->}{d}{g}
        \ncline{->}{d}{h}
        \ncline{->}{e}{i}
        \ncline{->}{e}{h}
        \ncline{->}{f}{i}
        \ncline{->}{f}{j}
        \ncline{->}{h}{k}
        \ncline{->}{i}{k}
        
        \psccurve[fillstyle=none, linestyle=dashed, linecolor=areaColor1](1, 7.5)(4, 8)(7, 7.5)(7, 6.5)(6, 6.5)(4.5, 5)(3,
        2)(0, 2)(1, 5.5)
  
        \psccurve[fillstyle=none, linestyle=dashed, linecolor=areaColor2](3, 1)(3.5, 2)(5, 4)(6, 6)(8, 3)(6, 2)(5, 0)
      \end{pspicture}
    }
    \subfloat[The encoding of the graph.]{
      \raisebox{4.5cm}{
      \begin{tabular}{c|cccc}
        \toprule
        & $id(t)$ & $I_i(t)$ & $s_0$ & $s_1$ \\ \midrule
        A & 2 & 0 & $[2;2]$ & $\emptyset$ \\
        B & 5 & 0 & $[5;5]$ & $\emptyset$ \\
        C & 3 & 0 & $[3;3]$ & $\emptyset$ \\
        D & 1 & 0 & $[1;3]$ & $\emptyset$ \\
        E & 4 & 0 & $[2;5]$ & $\emptyset$ \\
        F & 2 & 1 & $[3;3]$ & $[2;2]$ \\
        G & 0 & 0 & $[0;3]$ & $\emptyset$ \\
        H & 6 & 0 & $[1;6]$ & $\emptyset$ \\
        I & 1 & 1 & $[2;5]$ & $[1;2]$ \\
        J & 3 & 1 & $[3;3]$ & $[2;3]$ \\
        K & 0 & 1 & $[1;6]$ & $[0;2]$ \\ \bottomrule
      \end{tabular}
      }
    }
  \caption{A hierarchy encoded with the slicing method - the sample from \cite{baehni07}}
  \label{fig:slicing}
\end{figure}

%
% OPIS KAWAŁKA KODU

\begin{description}
\item[Testing for the existence of a path] \hfill \\ \code{isReachable(G, x, y)} \\ 
Given a digraph $G(V,A)$ and an ordered pair of vertices $(x,y)$ belonging to $G$, we have to check if there is a path
from $x$ to $y$. Formally: 
$$ \code{isReachable(G, x, y): Boolean} =  
\begin{cases}
  \text{true}  & \quad \text{if} \quad x \leq y \\
  \text{false} & \quad \text{otherwise}
\end{cases} $$
When we choose the classical method of graph representation, we have to perform a search procedure (either depth first
search or breadth first search), starting at the vertex $x$ (Fig. \ref{fig:checkingExistenceOfPath}). The extreme case
would be searching across all the vertices in the graph to get a result. In ontologies, reachability testing is done to
check whether a given object is the instance of a given concept or whether a given concept is a subconcept of another.

\end{description}


%
% Algorytm z komentarzem
%

\begin{algorithm}
  \caption{Scoring by the number of removed conflicts}
  \label{alg:scoringByNumOfRemovedConflicts}
  \begin{algorithmic}
    \State $TC \gets$ transitive closure of $G$
    \State $\delta_1, \delta_2 \gets$ primary dimensions
    \State $i \gets 2$
    \While {unresolved conflicts exist}
      \State $\delta_{i+1} \gets$ new dimension \Comment do not add it to the index
      \State $b \gets$ $bits(\delta_{i+1})$
      \State $c \gets$ resolve conflicts by genes with use of at most $b$ bits, and return the number of resolved
      conflicts
      \If {$resolvedConflicts(\delta_{i+1}) \geq c$} 
        \State undo the last step in the gene-based index
        \State add $\delta_{i+1}$ to the topological index
        \State $i \gets i+1$
      \EndIf
    \EndWhile   
  \end{algorithmic}
\end{algorithm}


% listing
\begin{lstlisting}[language=Java,style=outcode,label={lst:txState},caption={Transaction State data structure}]
class TransactionState
{
    UUID id
    Collection<TransactionItem> transactionItems    
}

class TransactionItem
{
    String tableName
    Token token
}
\end{lstlisting}
